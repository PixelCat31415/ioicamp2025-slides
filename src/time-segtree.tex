\newcommand{\InlineCode}[1]{\mintinline{cpp}|#1|}

\section{時間線段樹}

\iffalse

\begin{frame}{\ebtitle}
    如果要一句話解釋時間線段樹:\\
    「時間線段樹是一種用 stack 模擬 set 的辦法」

    \only<2> {
        (沒有解釋到半點 \emoji{pensive})
    }
\end{frame}

\fi

\begin{frame}{\ebtitle}
    \only<1,3> {
        時間線段樹想解決「操作有時效性」的問題
        
        你想要在未來的某個時間取消這次操作 \\
        但資料結構只有支援你做新的操作、和取消\yum{上一次}操作
    }

    \only<2> {
        例如:
        \begin{itemize}
            \item 你有一個 stack,想要加新元素、和移除任意久以前加進去的東西。但是 stack 只支援移除\yum{上一次}加進去的元素
            \item 你有一個併查集,想要加新的邊、和移除任意一條加過的邊。但是併查集只支援刪除\yum{上一次}加進去的邊
        \end{itemize}
    }

    \only<3> {
        如果\yum{操作沒有順序性}的話\\
        可以以複雜度乘 $O(\log Q)$ 的代價,使用時間線段樹支援 $Q$ 次這類操作
    }
\end{frame}

\begin{frame}{\ebtitle}
    線段樹的實做框架(講義第 79 頁)

    \begin{itemize}
        \item 先收集每個操作的有效時間段,加到時間線段樹裡
        \item 遍歷時間線段樹重現所有操作
    \end{itemize}
\end{frame}

\begin{frame}{\ebtitle}
    \begin{problem}[【Gate】這個笑容由我來守護 - EXTREME,TIOJ 1903]
        給定一張 $N$ 個點的無向圖,一開始圖上有 $M$ 條邊。

        現在有 $Q$ 個操作,每個操作會是以下兩種中的一種:
        
        \begin{itemize}
            \item 增加一條連接著編號 $A_i$ 與編號 $B_i$ 的邊。
            \item 刪除一條連接著編號 $A_i$ 與編號 $B_i$ 的邊(保證這條邊是存在的)。
        \end{itemize}

        每次操作完後請輸出當前的連通塊數量。

        \begin{itemize}
            \item $1\le N\le 5\times 10^5$。
            \item $Q\le 5\times 10^5$。
        \end{itemize}
    \end{problem}
\end{frame}

\begin{frame}{\ebtitle}
    \begin{itemize}
        \item 時間線段樹對「操作時間」開一棵線段樹
        \item 每個節點存一些操作
        \item 事先收集所有操作,在線段樹上的一些節點存起來
        \item \InlineCode{traversal} 遍歷線段樹,重現所有操作
    \end{itemize}
\end{frame}

\begin{frame}{\ebtitle}
    如果開始到結束經歷 $Q$ 次加邊或刪邊 \\
    對這 $Q$ 個時間點開線段樹 \\
    每個葉子的所有祖先存的所有邊,會是這個時間點應該要活著的所有邊
\end{frame}

\begin{frame}[fragile]{\ebtitle}
    加入操作:像區間修改一樣,把要加的邊紀錄在被修改到的節點上
    \begin{minted}[linenos=false]{cpp}
        void insert_event(
            int idx, int lb, int rb, int ql, int qr, Event e
        ) {
            if (ql <= lb && rb <= qr) {
                tr[idx].push_back(e);
                return;
            }   
            int mid = (lb + rb) / 2;
            if(ql <= mid)
                insert_event(idx * 2, lb, mid, ql, qr, e);
            if(qr > mid)
                insert_event(idx * 2 + 1, mid + 1, rb, ql, qr, e);
        }
    \end{minted}
\end{frame}

\begin{frame}[fragile]{\ebtitle}
    遍歷線段樹:進入節點時把剛剛加的邊加進 DSU,離開節點時移除這些邊
    \begin{minted}[linenos=false]{cpp}
        void traversal(int idx, int lb, int rb) {
            int cnt = 0;
            for (auto e : tr[idx])
                if (do_things(e))
                    cnt++;
            if (lb == rb) {
                // 記錄在這個時間點的答案
            } else {
                int mid = (lb + rb) / 2;
                traversal(idx * 2, lb, mid);
                traversal(idx * 2 + 1, mid + 1, rb); 
            }
            while (cnt--) undo();
        }
    \end{minted}
\end{frame}

% \begin{frame}{\ebtitle}
%     \todo
% \end{frame}

\begin{frame}{\ebtitle}
    \begin{itemize}
        \item DFS 的時候每個節點在最後剛好把自己加進去的邊拿掉
        \item 葉子代表一個時間點,每次 DFS 到葉子的時候,DSU 裡面正好有所有應該活著的邊
    \end{itemize}
\end{frame}

\begin{frame}{\ebtitle}
    最後一個技術困難:DSU 要怎麼取消上一次操作?

    \begin{itemize}
        \item 每次 DSU 加邊只會把一個節點接到另一個身上
        \item 紀錄每次加邊是誰連到誰,undo 的時候還原這兩個節點的狀態
        \item 啟發式合併 \emoji{check mark button}
        \item 路徑壓縮 \emoji{cross mark}
    \end{itemize}
\end{frame}

\begin{frame}{\ebtitle}
    Recap
    
    \begin{itemize}
        \item 有一個可以 undo 的併查集
        \item 收集每一條邊存活的時間,加入時間線段樹
        \item 遍歷時間線段樹,一邊紀錄答案
    \end{itemize}
\end{frame}

\begin{frame}{\ebtitle}
    時間複雜度
    
    \begin{itemize}
        \item 總共出現過 $O(M + Q)$ 條邊
        \item 每條邊在線段樹上被拆成 $O(\log Q)$ 個操作
        \item 每個操作要在併查集上 $O(\log N)$ 加邊、$O(1)$ undo
    \end{itemize}

    總複雜度 $O((M + Q) \log Q \log N)$ \\
    比只有加邊的啟發式合併併查集多一個 $\log$
\end{frame}

\begin{frame}{\ebtitle}
    使用時間線段樹的場合
    
    \begin{itemize}
        \item 操作有時效性
        \item 操作可以交換
        \item 可以反悔上一個操作
    \end{itemize}
\end{frame}

% \subsection{時間線段樹與分治}

\begin{frame}{\ectitle}
    線段樹對應某種分治演算法的遞迴樹

    時間線段樹是什麼東西的遞迴樹?
\end{frame}

\begin{frame}{\ectitle}
    \begin{problem}[Arctan Loves Stacks,台大演算法設計與分析(ADA)作業]
        你想用一個 \InlineCode{std::stack}(\InlineCode{push/pop})來模擬 $N$ 個 \InlineCode{std::set} 操作(\InlineCode{insert/erase}) \\

	    $N$ 的值和所有 set 操作都是一開始就事先知道的,請用總共 $O(N \log N)$ 次 stack 操作來模擬 $N$ 個 set 操作
    \end{problem}
\end{frame}

\newcommand{\CR}{\emoji{red circle}}
\newcommand{\CB}{\emoji{blue circle}}
\newcommand{\OK}{\emoji{check mark button}}
\begin{frame}{\ectitle}
    \begin{multicols}{2}
        Set 操作:
        \begin{table}
            \begin{tabular}{| l p{0.2cm} l |}
                \hline
                insert \CR  && $\{\CR\}$ \\
                insert \CB && $\{\CR, \CB\}$ \\
                erase \CR   && $\{\CB\}$ \\
                erase \CB  && $\{\}$ \\
                \hline
            \end{tabular}
        \end{table}
        \columnbreak
        \only<1> { \hspace{5cm} }
        \only<2> {
            Stack 操作:
            \begin{table}
                \begin{tabular}{| l p{0.2cm} l l |}
                    \hline
                    push \CR && $\{\CR\}$      & \OK \\
                    push \CB && $\{\CR, \CB\}$ & \OK \\
                    pop  \CB && $\{\CR\}$      &     \\
                    pop  \CR && $\{\}$         &     \\
                    push \CB && $\{\CB\}$      & \OK \\
                    pop  \CB && $\{\}$         & \OK \\
                    \hline
                \end{tabular}
            \end{table}
        }
    \end{multicols}
\end{frame}

\begin{frame}{\ectitle}
    \begin{itemize}
        \item insert $\CR_1, \CR_2, \CR_3, \CR_4, \CR_5$
        \item insert $\CB_1, \CB_2, \CB_3, \CB_4, \CB_5$
    \end{itemize}

    Stack:$\{\CR_1, \CR_2, \CR_3, \CR_4, \CR_5, \CB_1, \CB_2, \CB_3, \CB_4, \CB_5\}$

    \begin{itemize}
        \item erase $\CR_1, \CR_2, \CR_3, \CR_4, \CR_5$
    \end{itemize}

    每次都把所有東西拿出來再放回去?
\end{frame}

\begin{frame}{\ectitle}
    目標是只用 $O(N \log N)$ 次 stack 操作完成模擬 \\
    需要根據 set 操作設計合理的操作順序
\end{frame}

\begin{frame}{\ectitle}
    % todo
\end{frame}
