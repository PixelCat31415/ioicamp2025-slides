\section{線段樹的暴力與懶人標記}

\begin{frame}{\ebtitle}
    Change my mind:均攤分析就是玄學
\end{frame}

\begin{frame}{\btitle{區間開根號}}
    \begin{problem}[帶修改區間和,Zerojudge c652]
        給你一段 $N$ 個正整數的序列 $a_1,\cdots,a_N$,請你執行 $Q$ 筆操作。

        \begin{itemize}
            \item $0\;l\;r$:代表詢問 $[l,r]$ 區間的和。
            \item $1\;l\;r$:代表將 $[l,r]$ 區間的每個數字 $a_i$ 改成 $\lfloor{\sqrt{a_i}}\rfloor$。
        \end{itemize}

        \begin{itemize}
            \item $1\le N,Q\le 3\times 10^5$。
            \item $1\le a_i\le 10^{12}$。
        \end{itemize}
    \end{problem}
\end{frame}

\begin{frame}{\btitle{區間開根號}}
    如果想在線段樹維護區間和 \\
    區間開根號的時候區間和會如何變化?

    \only<2> {
        我也不知道 \emoji{sob}

        區間開根號沒有\strong{可預測性},不能打懶人標記
    }
\end{frame}

\begin{frame}{\btitle{區間開根號}}
    觀察:一個數字被開 $\log \log C$ 次根號之後就不會再動了,永遠都會是 $1$

    \only<2> {
        \begin{itemize}
            \item 如果區間內有人不是 $1$,暴力往下修改
            \item 如果區間內所有人都是 $1$,什麼事都不需要做
        \end{itemize}

        一個數字只會被暴力改 $\log \log C$ 次,每次 $O(\log N)$ 時間 \\
        總時間複雜度 $O(Q \log N + N \log N \log\log C)$
    }
\end{frame}

\begin{frame}{\btitle{區間開根號.其二}}
    \begin{problem}[帶修改區間和 Ex.,波路自編題]
        給你一段 $N$ 個正整數的序列 $a_1,\cdots,a_N$,請你執行 $Q$ 筆操作。

        \begin{itemize}
            \item \yum{$1\;l\;r\;c$:代表將 $[l,r]$ 區間的每個數字 $a_i$ 加上 $c$。}
            \item $2\;l\;r$:代表將 $[l,r]$ 區間的每個數字 $a_i$ 改成 $\lfloor{\sqrt{a_i}}\rfloor$。
            \item $3\;l\;r$:代表詢問 $[l,r]$ 區間的和。
        \end{itemize}

        \begin{itemize}
            \item $1\le N,Q\le 10^5$。
            \item $0\le a_i,c\le 10^9$。
        \end{itemize}
    \end{problem}
\end{frame}

\begin{frame}{\btitle{區間開根號.其二}}
    開完根號再加值,一個數字只會被暴力改......$O(Q)$ 次(???)

    剛剛的作法壞掉了
\end{frame}

\begin{frame}{\btitle{區間開根號.其二}}
    \only<1> {
        觀察:區間全距被開幾次根號之後就幾乎不動了
    }

    \only<2> {
        觀察:區間全距被開 \yum{$O(\log\log C)$} 次根號之後就會一直是 \yum{$0$ 或 $1$}

        對全距是 $1$ 的區間開根號可以打懶人標記嗎?\\
        可以,多紀錄最小值和最小值個數就知道區間裡有哪些數字

        線段樹修改時,只要全距還不是 $1$ 就暴力往下修改
    }
\end{frame}

\begin{frame}{\btitle{區間開根號.其二}}
    一開始,每個節點的全距都是 $O(C)$,暴力往下次數 $O(N \log\log C)$

    每次區間加值讓 $O(\log N)$ 個節點的全距增加 $O(C)$,暴力往下次數增加 $O(\log N \log\log C)$

    總時間複雜度 $O(Q \log N + N \log\log C + Q \log N \log\log C)$
\end{frame}

\begin{frame}[fragile]{\ebtitle}
    \begin{minted}[linenos=false]{cpp}
        void modify(int node, int l, int r, int ql, int qr) {
            if(l >= ql && r <= qr) {
                give_tag(node); return;
            }
            push(node);
            int m = (l + r) / 2;
            if(ql <= m) modify(L(node), l, m, ql, qr);
            if(qr > m)  modify(R(node), m + 1, r, ql, qr);
            pull(node);
        }
    \end{minted}
\end{frame}

\begin{frame}[fragile]{\ebtitle}
    \begin{minted}[linenos=false]{cpp}
        void modify(int node, int l, int r, int ql, int qr) {
            if(全距 <= 1) {
                give_tag(node); return;
            }
            push(node);
            int m = (l + r) / 2;
            if(ql <= m) modify(L(node), l, m, ql, qr);
            if(qr > m)  modify(R(node), m + 1, r, ql, qr);
            pull(node);
        }
    \end{minted}
\end{frame}

\begin{frame}[fragile]{\ebtitle}
    \begin{minted}[linenos=false]{cpp}
        void modify(int node, int l, int r, int ql, int qr) {
            if(tag_condition(node)) {
                give_tag(node); return;
            }
            push(node);
            int m = (l + r) / 2;
            if(ql <= m) modify(L(node), l, m, ql, qr);
            if(qr > m)  modify(R(node), m + 1, r, ql, qr);
            pull(node);
        }
    \end{minted}
\end{frame}

\begin{frame}{\ebtitle}
    也許......\mintinline{cpp}|tag_condition| 還可以是......?
\end{frame}

\begin{frame}{\btitle{Segment Tree Beats}}
    \begin{problem}[Gorgeous Sequence,HDU 5306]
        $T$ 筆測資,每筆測資給你一段 $N$ 個整數的序列 $a_1,\cdots,a_N$,請你執行 $Q$ 筆操作。

        \begin{itemize}
            \item $0\;l\;r\;t$:代表將 $[l,r]$ 區間的每個數字 $a_i$ 改成 $\min(a_i,t)$。
            \item $1\;l\;r$:代表詢問 $[l,r]$ 區間的最大值。
            \item $2\;l\;r$:代表詢問 $[l,r]$ 區間的和。
        \end{itemize}
        
        \begin{itemize}
            \item $1\le T\le 100$。
            \item $1\le \sum N,\sum Q\le 10^6$。
            \item $0\le a_i,t< 2^{31}$。
        \end{itemize}
    \end{problem}

    區間取 min 對區間和同樣不能預測,不能直接打懶人標記
\end{frame}

\begin{frame}{\btitle{Segment Tree Beats}}
    每個節點維護\yum{區間嚴格次大值}\only<2>{ \brilliance{brilliant} }和最大值個數

    \only<2-> {
        \begin{itemize}
            \item 如果 $t \leq$ 次大值,暴力往下修改
            \item 如果 $t >$ 次大值,等同於把所有最大值都改成 $t$,可以打懶人標記
        \end{itemize}

        時間複雜度:$O((N + Q) \log N)$\only<3>{???\\}
        \only<3>{憑什麼這麼快?}
    }
\end{frame}

\begin{frame}{\btitle{Segment Tree Beats}}
    考慮每個節點的數字種類數 \\
    每次往下暴力修改,額外花 $O(1)$ 時間,區間內的數字一定會少至少一種

    比一般線段樹多付出的時間 \\
    最多是每個節點暴力往下修改的總次數 \\
    也就是 $O(N \log N)$

    \only<2> {
        總時間複雜度 $O((N + Q) \log N)$
    }
\end{frame}

\begin{frame}{\btitle{Segment Tree Beats.其二}}
    \begin{problem}
        給你一段 $N$ 個整數的序列 $a_1,\cdots,a_N$,請你執行 $Q$ 筆操作。

        \begin{itemize}
            \item $0\;l\;r\;t$:代表將 $[l,r]$ 區間的每個數字 $a_i$ 改成 $\min(a_i,t)$。
            \item \yum{$1\;l\;r\;c$:代表將 $[l,r]$ 區間的每個數字加上 $c$。}
            \item $2\;l\;r$:代表詢問 $[l,r]$ 區間的最大值。
            \item $3\;l\;r$:代表詢問 $[l,r]$ 區間的和。
        \end{itemize}

        \begin{itemize}
            \item $1\le N,Q\le 3\times 10^5$。
            \item $-10^6\le c,a_i,t\le 10^6$。
        \end{itemize}
    \end{problem}
\end{frame}

\begin{frame}{\btitle{Segment Tree Beats.其二}}
    嘗試跟前一題用一樣的作法

    區間加值後,節點的數字種類數會變多......\\
    \only<2->{$O(\text{區間長度})$}

    \only<2->{沿用相同的證明想法,暴力修改的次數最多是......\\}
    \only<3->{$O(NQ)$?}

    \only<4>{
        換一種證明思路,可以證明總複雜度是 $O((N + Q) \log^2 N)$ 的
    }
\end{frame}

\begin{frame}{\btitle{Segment Tree Beats.其二}}
    \todo 插圖
\end{frame}

\begin{frame}{\btitle{Segment Tree Beats.其二}}
    「$t \leq$ 區間次小值時,往下暴力」\\
    實際上等同往下 DFS 移除子樹內 $\geq t$ 的標記

    \only<2> {
        移除一個標記要花 $O(\text{樹高}) = O(\log N)$ 時間

        一開始最多有 $N$ 個標記 \\
        什麼時候標記會變多?
    }
\end{frame}

\begin{frame}{\btitle{Segment Tree Beats.其二}}
    在線段樹上區間操作的時候,可以把節點分成四種

    \begin{enumerate}[A]
        \item 被操作區間完全包含
        \item 跟操作區間部份重疊
        \item 跟操作區間不重疊,但是 B 的子節點
        \item 跟操作區間不重疊的其他節點
    \end{enumerate}
\end{frame}

\begin{frame}{\btitle{Segment Tree Beats.其二}}
    \todo 插圖
\end{frame}

\begin{frame}{\btitle{Segment Tree Beats.其二}}
    \only<1> {
        標記變多例:A 類節點獲得標記(被操作區間完全包含)
        % 1, 2, 2, 2 --{[4, 4] chmin 1}-> 1, 2, 2, 1
    }

    \only<2> {
        標記變多例:B 類節點獲得標記(跟操作區間部份重疊)
        % 1, 2, 2, 2 --{[2, 3] chmin 1}-> 1, 1, 1, 2
    }

    \only<3> {
        標記變多例:C 類節點獲得標記(跟操作區間不重疊,但是 B 的子節點)
        % 1, 2, 2, 2 --{[2, 3] add 1}-> 1, 3, 3, 2
    }

    \only<4> {
        標記變多例:D 類節點獲得標記(跟操作區間不重疊的其他節點)
        
        並不會,因為節點和父節點內的最大值都沒有變
    }
\end{frame}

\begin{frame}{\btitle{Segment Tree Beats.其二}}
    場合 1:區間 chmin

    \begin{enumerate}[A]
        \item 被操作區間完全包含
        \begin{itemize}
            \item 減少若干個標記
            \item 增加至多 $O(\log N)$ 個標記
        \end{itemize}
        \item 跟操作區間部份重疊
        \begin{itemize}
            \item 至多 $O(\log N)$ 個節點
        \end{itemize}
        \item 跟操作區間不重疊,但是 B 的子節點
        \begin{itemize}
            \item 至多 $O(\log N)$ 個節點
        \end{itemize}
        \item 跟操作區間不重疊的其他節點
        \begin{itemize}
            \item 標記維持原狀
        \end{itemize}
    \end{enumerate}

    標記最多增加 $O(\log N)$ 個
\end{frame}

\begin{frame}{\btitle{Segment Tree Beats.其二}}
    場合 2:區間加值

    \begin{enumerate}[A]
        \item 被操作區間完全包含
        \begin{itemize}
            \item 標記維持原狀
        \end{itemize}
        \item 跟操作區間部份重疊
        \begin{itemize}
            \item 至多 $O(\log N)$ 個節點
        \end{itemize}
        \item 跟操作區間不重疊,但是 B 的子節點
        \begin{itemize}
            \item 至多 $O(\log N)$ 個節點
        \end{itemize}
        \item 跟操作區間不重疊的其他節點
        \begin{itemize}
            \item 標記維持原狀
        \end{itemize}
    \end{enumerate}

    標記最多增加 $O(\log N)$ 個
\end{frame}

\begin{frame}{\btitle{Segment Tree Beats.其二}}
    \begin{itemize}
        \item 「暴力往下」實際上是在 DFS 刪除標記
        \item 「暴力往下」刪除一個標記花 $O(\log N)$ 時間
        \item 總共只有 $O(N + Q \log N)$ 個標記可以刪
    \end{itemize}

    所以,吉如一線段樹和一般線段樹相比,額外花的時間頂多只有 $O((N + Q) \log^2 N)$
\end{frame}

\begin{frame}{\btitle{Segment Tree Beats.其二}}
    吉如一本人給的證明和網路上流傳的證明都說可以 $O((N + Q) \log^2 N)$

    實際上執行飛快,被懷疑其實只有一個 $\log$
\end{frame}

\begin{frame}{\btitle{Segment Tree Beats.其三}}
    \begin{problem}[Range Chmin Chmax Add Range Sum,Library Checker]
        給你一段 $N$ 個整數的序列 $a_1,\cdots,a_N$,請你執行 $Q$ 筆操作。

        \begin{itemize}
            \item $0\;l\;r\;t$:代表將 $[l,r]$ 區間的每個數字 $a_i$ 改成 $\min(a_i,t)$。
            \item $1\;l\;r\;t$:代表將 $[l,r]$ 區間的每個數字 $a_i$ 改成 $\max(a_i,t)$。
            \item $2\;l\;r\;c$:代表將 $[l,r]$ 區間的每個數字加上 $c$。
            \item $3\;l\;r$:代表詢問 $[l,r]$ 區間的和。
        \end{itemize}
        
        \begin{itemize}
            \item $1\le N,Q\le 3\times 10^5$。
            \item $-10^6\le c,a_i,t\le 10^6$。
        \end{itemize}
    \end{problem}
\end{frame}

\begin{frame}{\btitle{Segment Tree Beats.其三}}
    加上了區間取 max 操作

    沿用同樣的作法同樣的證明,維護

    \begin{itemize}
        \item 區間最大、最小值
        \item 區間最大、最小值個數
        \item 區間\strong{嚴格}次大、次小值
    \end{itemize}

    時間複雜度 $O((N + Q) \log^2 N)$
\end{frame}

\begin{frame}{\btitle{Bear and Bad Powers of 42}}
    \begin{problem}[Bear and Bad Powers of 42,Codeforces 679E]
        給你一段 $N$ 個正整數的序列 $a_1,\cdots,a_N$,一個數字是好的若且唯若他不是 $42$ 的冪次,請你執行 $Q$ 筆操作。

        \begin{itemize}
            \item $1\;i$:輸出 $a_i$。
            \item $2\;l\;r\;x$:代表將 $[l,r]$ 區間的每個數字 $a_i$ 改成 $x$,保證 $x$ 是好的。
            \item $3\;l\;r\;c$:代表將 $[l,r]$ 區間的每個數字加上 $c$,並重複該操作直到 $[l,r]$ 區間的每個數字都是好的為止。
        \end{itemize}

        注意到每次操作後,所有數字都會是好的。

        \begin{itemize}
            \item $1\le N,Q\le 10^5$。
            \item $2\le a_i,x\le 10^9$。
            \item $1\le c\le 10^9$。
        \end{itemize}
    \end{problem}
\end{frame}

\begin{frame}{\btitle{Bear and Bad Powers of 42}}
    如果沒有區間改值,
    
    \begin{itemize}
        \item 一個數字頂多被加到 $NQ = 10^{14}$ 左右,而 $10^{14}$ 以內的 $42$ 冪次只有 $\log_{42} 10^{14}$ 不到十個
        \item 維護每個數字離下一個 $42$ 冪次還有多遠
        \item 時間複雜度 $O((N + Q) \log N \log_{42} NQ)$
    \end{itemize}
\end{frame}

\begin{frame}{\btitle{Bear and Bad Powers of 42}}
    如果加上區間改值,

    \only<1> {
        \begin{itemize}
            \item 不能暴力到底?暴力到什麼時候為止?
        \end{itemize}
    }
    
    \only<2-> {
        \begin{itemize}
            \item 暴力到\strong{區間內數字都一樣}為止
            \item 參考吉如一線段樹的證明,需要暴力很多次的節點不會增加很多
            \item 時間複雜度 $O((N + Q) \log N \log_{42} NQ)$\only<3>{\yum{???}}
        \end{itemize}
    }
\end{frame}

\begin{frame}{\btitle{Bear and Bad Powers of 42}}
    題外話:官解

    \begin{itemize}
        \item 被區間改值的那段數字視為「一坨」
        \item 區間操作的時候可能把一坨切成兩坨
        \item 一整坨可以一起加值
        \item 時間複雜度 $O((N + Q) \log N \log_{42} NQ)$
    \end{itemize}
\end{frame}

\begin{frame}{\ebtitle}
    Change my mind:均攤分析就是玄學
\end{frame}

\begin{frame}{\btitle{總結}}
    這不是一堂資料結構課,這是一堂\strong{均攤分析}課 \\
    資料結構不是重點,重點是均攤的思路、直覺、證明手法

    也許你此生沒機會真的砸吉如一線段樹,但均攤分析值得你學習
\end{frame}
