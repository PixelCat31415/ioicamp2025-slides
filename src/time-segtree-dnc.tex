\subsection{時間線段樹與分治}

\begin{frame}{\ectitle}
    線段樹對應某種分治演算法的遞迴樹

    時間線段樹是什麼東西的遞迴樹?
\end{frame}

\begin{frame}{\ectitle}
    \begin{problem}[Arctan Loves Stacks,台大演算法設計與分析(ADA)作業]
        你想用一個 \InlineCode{std::stack}(\InlineCode{push/pop})來模擬 $N$ 個 \InlineCode{std::set} 操作(\InlineCode{insert/erase}) \\

	    $N$ 的值和所有 set 操作都是一開始就事先知道的,請用總共 $O(N \log N)$ 次 stack 操作來模擬 $N$ 個 set 操作
    \end{problem}
\end{frame}

\newcommand{\CR}{\emoji{red circle}}
\newcommand{\CB}{\emoji{blue circle}}
\newcommand{\OK}{\emoji{check mark button}}
\begin{frame}{\ectitle}
    \begin{multicols}{2}
        Set 操作:
        \begin{table}
            \begin{tabular}{| l p{0.2cm} l |}
                \hline
                insert \CR  && $\{\CR\}$ \\
                insert \CB && $\{\CR, \CB\}$ \\
                erase \CR   && $\{\CB\}$ \\
                erase \CB  && $\{\}$ \\
                \hline
            \end{tabular}
        \end{table}
        \columnbreak
        \only<1> { \hspace{5cm} }
        \only<2> {
            Stack 操作:
            \begin{table}
                \begin{tabular}{| l p{0.2cm} l l |}
                    \hline
                    push \CR && $\{\CR\}$      & \OK \\
                    push \CB && $\{\CR, \CB\}$ & \OK \\
                    pop  \CB && $\{\CR\}$      &     \\
                    pop  \CR && $\{\}$         &     \\
                    push \CB && $\{\CB\}$      & \OK \\
                    pop  \CB && $\{\}$         & \OK \\
                    \hline
                \end{tabular}
            \end{table}
        }
    \end{multicols}
\end{frame}

\begin{frame}{\ectitle}
    \begin{itemize}
        \item insert $\CR_1, \CR_2, \CR_3, \CR_4, \CR_5$
        \item insert $\CB_1, \CB_2, \CB_3, \CB_4, \CB_5$
    \end{itemize}

    Stack:$\{\CR_1, \CR_2, \CR_3, \CR_4, \CR_5, \CB_1, \CB_2, \CB_3, \CB_4, \CB_5\}$

    \begin{itemize}
        \item erase $\CR_1, \CR_2, \CR_3, \CR_4, \CR_5$
    \end{itemize}

    每次都把所有東西拿出來再放回去?
\end{frame}

\begin{frame}{\ectitle}
    目標是只用 $O(N \log N)$ 次 stack 操作完成模擬 \\
    需要根據 set 操作設計合理的操作順序
\end{frame}

\begin{frame}{\ectitle}
    % todo
\end{frame}