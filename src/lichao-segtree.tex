\subsection{李超線段樹}

\begin{frame}{\ectitle}
    \begin{problem}[動態凸包]
        現在有 $Q$ 個操作,每個操作會是以下兩種中的一種:

        \begin{itemize}
            \item 加入一條直線 $y = mx + k$
            \item 詢問在 $x = t$ 處最大的 $y$ 值
        \end{itemize}

        \begin{itemize}
            \item $1\le Q \le 10^5$
            \item $|m|, |k| \le 10^9$
            \item $1 \le t \le 10^5$
        \end{itemize}
    \end{problem}
\end{frame}

\begin{frame}{\ectitle}
    用 set 維護上凸包上的線段,維護線段控制的左右界,每次加入直線先搜他控制的區間,往左右殺掉其他線段,查詢的時候二分搜是哪條線段代值進去。注意 iterator 使用、全整求線交點......

    太麻煩了,而且常數不小\sout{,而且我沒寫過} \\
    有沒有簡單一點的辦法?
\end{frame}

\begin{frame}{\ectitle}
    李超線段樹:
    \begin{itemize}
        \item 對要查詢的\yum{值域}開線段樹,葉子代表單一一個 $x$ 的值
        \item 每個節點存\yum{一條}對\yum{中點}來說 $y$ 最大的直線
        \begin{itemize}
            \item 對中點一定是有用的
            \item 可能還對這個區間的其他一部分有用
        \end{itemize}
    \end{itemize}
\end{frame}

\begin{frame}{\ctitle{插入直線}}
    原本節點上有一條直線 \\
    這次詢問想插入另外一條直線

    一個節點只能存一條線
    誰要留下來?另一條線要去哪裡?
\end{frame}

\begin{frame}{\ctitle{插入直線}}
    \only<1> {
        \begin{centikz}
            \draw[color=gray, dashed] (-5, 0) -- (-5, 5);
            \draw[color=gray, dashed] ( 0, 0) -- ( 0, 5);
            \draw[color=gray, dashed] ( 5, 0) -- ( 5, 5);
            \node[color=gray, anchor=north] at (-5, -0.2) {$L$};
            \node[color=gray, anchor=north] at ( 0, -0.2) {$M = \frac{L + R}{2}$};
            \node[color=gray, anchor=north] at ( 5, -0.2) {$R$};
            \draw[color=black, very thick] plot[domain=-5:5] (\x,{0.3 * \x + 2.8});
            \draw[color=black, very thick] plot[domain=-5:5] (\x,{-0.2 * \x + 2.1});
        \end{centikz}
    }

    \only<2> {
        \begin{centikz}
            \draw[color=gray, dashed] (-5, 0) -- (-5, 5);
            \draw[color=gray, dashed] ( 0, 0) -- ( 0, 5);
            \draw[color=gray, dashed] ( 5, 0) -- ( 5, 5);
            \node[color=gray, anchor=north] at (-5, -0.2) {$L$};
            \node[color=gray, anchor=north] at ( 0, -0.2) {$M = \frac{L + R}{2}$};
            \node[color=gray, anchor=north] at ( 5, -0.2) {$R$};
            \draw[color=black, very thick] plot[domain=-5:5] (\x,{-0.2 * \x + 2.1});
            \draw[color=Lime, very thick] plot[domain=-5:5] (\x,{0.3 * \x + 2.8});
            \node[anchor=south west] at (0, 3) {\brilliance{best}};
        \end{centikz}
    }

    \only<3> {
        \begin{centikz}
            \draw[color=gray, dashed] (-5, 0) -- (-5, 5);
            \draw[color=gray, dashed] ( 0, 0) -- ( 0, 5);
            \draw[color=gray, dashed] ( 5, 0) -- ( 5, 5);
            \node[color=gray, anchor=north] at (-5, -0.2) {$L$};
            \node[color=gray, anchor=north] at ( 0, -0.2) {$M = \frac{L + R}{2}$};
            \node[color=gray, anchor=north] at ( 5, -0.2) {$R$};
            
            \draw[color=DarkSeaGreen, very thick] plot[domain=-5:-1.4] (\x,{-0.2 * \x + 2.1});
            \draw[color=Red, very thick] plot[domain=-1.4:5] (\x,{-0.2 * \x + 2.1});
            \node[anchor=south] at (-3.2, 2.84) {\brilliance{good}};
            \node[anchor=south] at (2.5, 1.7) {\brilliance{incorrect}};
            
            \draw[color=Lime, very thick] plot[domain=-5:5] (\x,{0.3 * \x + 2.8});
            \node[anchor=south west] at (0, 3) {\brilliance{best}};
        \end{centikz}
    }
\end{frame}

\begin{frame}{\ctitle{插入直線}}
    \only<1> {
        \begin{centikz}
            \draw[color=gray, dashed] (-5, 0) -- (-5, 5);
            \draw[color=gray, dashed] ( 0, 0) -- ( 0, 5);
            \draw[color=gray, dashed] ( 5, 0) -- ( 5, 5);
            \node[color=gray, anchor=north] at (-5, -0.2) {$L$};
            \node[color=gray, anchor=north] at ( 0, -0.2) {$M = \frac{L + R}{2}$};
            \node[color=gray, anchor=north] at ( 5, -0.2) {$R$};
            \draw[color=black, very thick] plot[domain=-5:5] (\x,{0.3 * \x + 2.1});
            \draw[color=black, very thick] plot[domain=-5:5] (\x,{-0.2 * \x + 2.8});
        \end{centikz}
    }

    \only<2> {
        \begin{centikz}
            \draw[color=gray, dashed] (-5, 0) -- (-5, 5);
            \draw[color=gray, dashed] ( 0, 0) -- ( 0, 5);
            \draw[color=gray, dashed] ( 5, 0) -- ( 5, 5);
            \node[color=gray, anchor=north] at (-5, -0.2) {$L$};
            \node[color=gray, anchor=north] at ( 0, -0.2) {$M = \frac{L + R}{2}$};
            \node[color=gray, anchor=north] at ( 5, -0.2) {$R$};
            \draw[color=black, very thick] plot[domain=-5:5] (\x,{0.3 * \x + 2.1});
            \draw[color=Lime, very thick] plot[domain=-5:5] (\x,{-0.2 * \x + 2.8});
            \node[anchor=south west] at (0, 2.8) {\brilliance{best}};
        \end{centikz}
    }

    \only<3> {
        \begin{centikz}
            \draw[color=gray, dashed] (-5, 0) -- (-5, 5);
            \draw[color=gray, dashed] ( 0, 0) -- ( 0, 5);
            \draw[color=gray, dashed] ( 5, 0) -- ( 5, 5);
            \node[color=gray, anchor=north] at (-5, -0.2) {$L$};
            \node[color=gray, anchor=north] at ( 0, -0.2) {$M = \frac{L + R}{2}$};
            \node[color=gray, anchor=north] at ( 5, -0.2) {$R$};

            \draw[color=Red, very thick] plot[domain=-5:1.4] (\x,{0.3 * \x + 2.1});
            \draw[color=DarkSeaGreen, very thick] plot[domain=1.4:5] (\x,{0.3 * \x + 2.1});
            \node[anchor=south] at (-2.5, 1.45) {\brilliance{incorrect}};
            \node[anchor=south] at (3.2, 3.16) {\brilliance{good}};

            \draw[color=Lime, very thick] plot[domain=-5:5] (\x,{-0.2 * \x + 2.8});
            \node[anchor=south west] at (0, 2.8) {\brilliance{best}};
        \end{centikz}
    }
\end{frame}

\begin{frame}{\ctitle{插入直線}}
    一條直線在中點輸掉之後不能直接扔掉,因為他還沒輸光,區間內某些 $x$ 的範圍可能還需要他

    在中點輸掉的話,一定也會在左右其中一邊輸光 \\
    只有其中一邊可能還會需要用到這條直線,遞迴把他交給線段樹上那半邊的子樹處置,另外半邊已經不需要他了

    到葉子還輸的話那這條線徹底不會被任何人需要
\end{frame}

\begin{frame}{\ctitle{插入直線}}
    \only<1> {
        \begin{centikz}
            \draw[color=gray, dashed] (-5, 0) -- (-5, 5);
            \draw[color=gray, dashed] ( 0, 0) -- ( 0, 5);
            \draw[color=gray, dashed] ( 5, 0) -- ( 5, 5);
            \node[color=gray, anchor=north] at (-5, -0.2) {$L$};
            \node[color=gray, anchor=north] at ( 0, -0.2) {$M = \frac{L + R}{2}$};
            \node[color=gray, anchor=north] at ( 5, -0.2) {$R$};
            \draw[color=black, very thick] plot[domain=-5:5] (\x,{0.3 * \x + 2.8});
            \draw[color=black, very thick] plot[domain=-5:5] (\x,{-0.2 * \x + 2.1});
        \end{centikz}
    }

    \only<2> {
        \begin{centikz}
            \draw[color=gray, dashed] (-5, 0) -- (-5, 5);
            \draw[color=gray, dashed] ( 0, 0) -- ( 0, 5);
            \draw[color=gray, dashed] ( 5, 0) -- ( 5, 5);
            \node[color=gray, anchor=north] at (-5, -0.2) {$L$};
            \node[color=gray, anchor=north] at ( 0, -0.2) {$M = \frac{L + R}{2}$};
            \node[color=gray, anchor=north] at ( 5, -0.2) {$R$};
            \draw[color=DarkSeaGreen, very thick] plot[domain=-5:0] (\x,{-0.2 * \x + 2.1});
            \draw[color=Lime, very thick] plot[domain=-5:5] (\x,{0.3 * \x + 2.8});
            \node[anchor=south west] at (0, 3) {\brilliance{best}};
            \node[anchor=south] at (-2.5, 2.7) {\brilliance{forced}};
        \end{centikz}
    }

    \only<3> {
        \begin{centikz}
            \draw[color=gray, dashed] (-5, 0) -- (-5, 5);
            \draw[color=gray, dashed] ( 0, 0) -- ( 0, 5);
            \draw[color=gray, dashed] ( 5, 0) -- ( 5, 5);
            \node[color=gray, anchor=north] at (-5, -0.2) {$L$};
            \node[color=gray, anchor=north] at ( 0, -0.2) {$M = \frac{L + R}{2}$};
            \node[color=gray, anchor=north] at ( 5, -0.2) {$R$};
            \draw[color=black, very thick] plot[domain=-5:5] (\x,{0.3 * \x + 2.1});
            \draw[color=black, very thick] plot[domain=-5:5] (\x,{-0.2 * \x + 2.8});
        \end{centikz}
    }

    \only<4> {
        \begin{centikz}
            \draw[color=gray, dashed] (-5, 0) -- (-5, 5);
            \draw[color=gray, dashed] ( 0, 0) -- ( 0, 5);
            \draw[color=gray, dashed] ( 5, 0) -- ( 5, 5);
            \node[color=gray, anchor=north] at (-5, -0.2) {$L$};
            \node[color=gray, anchor=north] at ( 0, -0.2) {$M = \frac{L + R}{2}$};
            \node[color=gray, anchor=north] at ( 5, -0.2) {$R$};
            \draw[color=DarkSeaGreen, very thick] plot[domain=0:5] (\x,{0.3 * \x + 2.1});
            \draw[color=Lime, very thick] plot[domain=-5:5] (\x,{-0.2 * \x + 2.8});
            \node[anchor=south west] at (0, 2.8) {\brilliance{best}};
            \node[anchor=south] at (2.5, 2.95) {\brilliance{forced}};
        \end{centikz}
    }
\end{frame}

\begin{frame}{\ctitle{插入直線}}
    從根節點出發,到葉節點為止:

    \begin{itemize}
        \item 代中點 $x$ 座標比較兩條直線,贏家留在節點上
        \item 比較兩條直線的斜率
        \begin{itemize}
            \item 如果贏家的斜率比較\strong{大},輸家往\strong{左}子樹遞迴插入
            \item 如果贏家的斜率比較\strong{小},輸家往\strong{右}子樹遞迴插入
        \end{itemize}
    \end{itemize}

    一直往子樹丟包直線\\
    時間複雜度 $O(\text{線段樹高}) = O(\log N)$
\end{frame}

\begin{frame}{\ctitle{單點查詢}}
    直線被扔到隔壁節點,代表這個範圍的 $x$ 全都用不到這條直線 \\
    一個 $x$ 可能用到的直線,都存在他的祖先們身上

    \begin{itemize}
        \item 找到代表這個 $x$ 值的葉子
        \item 檢查所有祖先存的直線,每個都代一次,回答最大的 $y$
    \end{itemize}

    時間複雜度 $O(\text{線段樹高}) = O(\log N)$
\end{frame}

\begin{frame}[fragile]{\ctitle{實做}}
    包裝直線作為函數使用
    
    \begin{minted}{cpp}
        struct Line {
            int a, b;  // y = ax + b
            Line(int _a = 0, int _b = 0): a(_a), b(_b) {}
            int operator()(int x) { return a * x + b; }
        };
    \end{minted}
\end{frame}

\begin{frame}[fragile]{\ctitle{實做}}
    插入直線

    \begin{itemize}
        \item 代中點 $x$ 座標比較兩條直線,贏家留在節點上
        \item 比較兩條直線的斜率
        \item 遞迴插入
    \end{itemize}
    
    \begin{minted}{cpp}
        void insert(int id, int l, int r, Line ln) {
            int m = (l + r) / 2;
            if(lns[id](m) < ln(m)) swap(lns[id], ln);
            if(l == r) return;
            if(lns[id].a > ln.a) insert(L(id), l, m, ln);
            else insert(R(id), m + 1, r, ln);
        }
    \end{minted}
\end{frame}

\begin{frame}[fragile]{\ctitle{實做}}
    單點查詢

    \begin{itemize}
        \item 找到代表這個 $x$ 值的葉子
        \item 檢查所有祖先存的直線,每個都代一次,回答最大的 $y$
    \end{itemize}
    
    \begin{minted}{cpp}
        int qry(int id, int l, int r, int x) {
            int m = (l + r) / 2;
            int res = lns[id](x);
            if(l == r) return res;
            if(x <= m) res = max(res, qry(L(id), l, m, x));
            else res = max(res, qry(R(id), m + 1, r, x));
            return res;
        }
    \end{minted}
\end{frame}

\begin{frame}{\ctitle{座標壓縮}}
    \begin{problem}[Line Add Get Min,Library Checker]
        你有 $N$ 條直線 $y = a_i x + b_i$。請你處理 $Q$ 個詢問:

        \begin{itemize}
            \item 加入一條直線 $y = ax + b$
            \item 詢問 $x = p$ 處最小的 $y$ 值
        \end{itemize}

        \begin{itemize}
            \item $1 \le N, Q \le 2 \times 10^5$
            \item \yum{$|a_i|, |p| \le 10^9$}
            \item $|b_i| \le 10^{18}$
        \end{itemize}
    \end{problem}
\end{frame}

\begin{frame}{\ctitle{座標壓縮}}
    剛剛對 $x$ 的值域 $1, 2, \dots, 10^5$ 開線段樹

    現在事先收集會被詢問到的 $x$ 座標 \\
    對會被問到的 $x$ 開線段樹

    詢問是浮點數的時候也可以
\end{frame}

\begin{frame}{\ctitle{座標壓縮}}
    如果事先不知道詢問位置呢?

    \only<2> {
        動態開點,用不到的節點不要理他
    }
\end{frame}

\begin{frame}{\ctitle{插入線段}}
    如果插入的不是直線,而是有左右範圍限制的線段呢?
    \begin{problem}[Segment Add Get Min,Library Checker]
        你有 $N$ 段\yum{線段} $y = a_i x + b_i$($x \in [l_i, r_i)$)。請你處理 $Q$ 個詢問:

        \begin{itemize}
            \item 加入一段\yum{線段} $y = ax + b$($x \in [l, r)$)
            \item 詢問 $x = p$ 處最小的 $y$ 值
        \end{itemize}

        \begin{itemize}
            \item $1 \le N, Q \le 2 \times 10^5$
            \item $-10^9 \leq l_i < r_i \leq 10^9$
            \item $|a_i|, |p| \le 10^9$
            \item $|b_i| \le 10^{18}$
        \end{itemize}
    \end{problem}
\end{frame}

\begin{frame}{\ctitle{插入線段}}
    \only<1> {
        \begin{centikz}
            \draw[color=gray, dashed] (-5, 0) -- (-5, 5);
            \draw[color=gray, dashed] ( 0, 0) -- ( 0, 5);
            \draw[color=gray, dashed] ( 5, 0) -- ( 5, 5);
            \node[color=gray, anchor=north] at (-5, -0.2) {$L$};
            \node[color=gray, anchor=north] at ( 0, -0.2) {$M = \frac{L + R}{2}$};
            \node[color=gray, anchor=north] at ( 5, -0.2) {$R$};
            \draw[color=black, very thick] plot[domain=-5:5] (\x,{-0.2 * \x + 2.1});
            \draw[color=black, very thick] plot[domain=-3:2] (\x,{0.3 * \x + 2.8});
        \end{centikz}
    }

    \only<2> {
        (往左右子樹都丟包線段一定是不行的)

        \begin{centikz}
            \draw[color=gray, dashed] (-5, 0) -- (-5, 5);
            \draw[color=gray, dashed] ( 0, 0) -- ( 0, 5);
            \draw[color=gray, dashed] ( 5, 0) -- ( 5, 5);
            \node[color=gray, anchor=north] at (-5, -0.2) {$L$};
            \node[color=gray, anchor=north] at ( 0, -0.2) {$M = \frac{L + R}{2}$};
            \node[color=gray, anchor=north] at ( 5, -0.2) {$R$};
            
            \draw[color=DarkSeaGreen, very thick] plot[domain=-5:-1.4] (\x,{-0.2 * \x + 2.1});
            \draw[color=Red, very thick] plot[domain=-1.4:2] (\x,{-0.2 * \x + 2.1});
            \draw[color=DarkSeaGreen, very thick] plot[domain=2:5] (\x,{-0.2 * \x + 2.1});
            \node[anchor=south] at (-3.2, 2.84) {\brilliance{good}};
            \node[anchor=south] at (1, 2) {\brilliance{incorrect}};
            \node[anchor=south] at (3.5, 1.5) {\brilliance{good}};
            
            \draw[color=Lime, very thick] plot[domain=-3:2] (\x,{0.3 * \x + 2.8});
            \node[anchor=south west] at (0, 3) {\brilliance{best}};
        \end{centikz}
    }
\end{frame}

\begin{frame}{\ctitle{插入線段}}
    一般線段樹是怎麼做區間修改的?
    \only<2-> {
        \\找 $O(\log N)$ 個節點覆蓋詢問的區間,修改那些節點
    }

    \only<3-> {
        找 $O(\log N)$ 個節點覆蓋線段範圍,對那些節點插入直線
    }

    \only<4-> {
        時間複雜度:插入一次 $O(\log N)$,總共 $O(\log^2N)$
    }
\end{frame}

\begin{frame}{\ctitle{應用}}
    \begin{itemize}
        \item 斜率優化 \emoji{white_check_mark}
        \item<2> 四邊形優化 \emoji{white_check_mark}
    \end{itemize}

    \only<2> {
        不只是直線,有\strong{優超性}的函數都可以
    }
\end{frame}

% todo: li-chao segtree extensions
