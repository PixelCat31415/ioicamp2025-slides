\section{Pattern}

\begin{frame}{\ebtitle}
    \begin{quote}
        資料結構往往不會赤裸出現 \\
    
        不是「我要用這個資結砸掉這題」\\
        而是「這題需要這樣做,所以可以拿這個資結砸掉」\\
        永遠都先想怎麼做,再尋找合適的資結幫助你 \\

        -- 2024 基礎資結
    \end{quote}

    \only<2> {
        有時候,題目要你維護的東西實在是太荒謬了,你需要自己創造可以維護的東西去維護
    }
\end{frame}

\begin{frame}{\btitle{Taxis}}
    \only<1,3-> {
        \begin{problem}[Taxis,POI 2018]
            現在有 $n$ 台計程車編號 $1\sim n$,對於第 $i$ 台計程車,給定 $s_i,c_i$,代表該台計程車的收費方式為 $s_i + d\times c_i$,其中 $d$ 為里程數。
    
            給定一個 $1\sim n$的排列,請問是否存在一個里程數 $x/y$,使得把所有計程車的編號照著收費由小到大列出\only<1-4>{恰好是這個排列}\only<5>{\yum{恰好是這個排列???}}(當收費價格一樣,順序可以任意安排),若存在的話請輸出任意一組解,否則輸出「NIE」。
    
            \only<1-3> {
                \only<1>{接著會有 $q$ 筆修改}\only<3>{\yum{接著會有 $q$ 筆修改}},每筆修改會有 $a_i,b_i$,代表交換排列在 $a_i$ 和 $b_i$ 的數字,每次交換後皆須輸出先前問題的答案。\only<3>{\yum{???}}
            }
    
            \begin{itemize}
                \item $1\le n\le 5\times10^5$。
                \item $1\le q\le 5\times10^5$。
            \end{itemize}
        \end{problem}
    }

    \only<2> {
        \todo 插圖
    }
\end{frame}

\begin{frame}{\btitle{Taxis}}
    什麼時候......

    \begin{itemize}
        \item \only<2->{\emoji{vomiting_face}} \sout<2->{計程車照收費排序剛好是 $1, 2, \dots, n$}
        \only<3-> {
            \item \only<4>{\emoji{smiling face with hearts}} $1$ 號車收費 $\leq 2$ 號,而且
            \item \only<4>{\emoji{smiling face with hearts}} $2$ 號車收費 $\leq 3$ 號,而且
            \item ......
            \item \only<4>{\emoji{smiling face with hearts}} $n - 1$ 號車收費 $\leq n$ 號
        }
    \end{itemize}

    \only<4> {
        $n = 2$ 我們總會做了吧?
    }
\end{frame}

\begin{frame}{\btitle{Taxis}}
    只考慮 $i$ 號車和 $i + 1$ 號車,可以滿足「$i$ 在 $i + 1$ 前面」的距離 $d$ 是一個 $d \leq \square$ 或 $d \geq \square$ 的限制

    當每一組相鄰計程車的順序都滿足,所有車就會照順序排好
    
    \only<2> {
        維護一坨射線有沒有交集 \\
        可以用兩個 multiset 維護
    }
\end{frame}

\begin{frame}{\btitle{Taxis}}
    帶修改?

    \sout<2->{「每筆修改會有 $a_i,b_i$,代表交換排列在 $a_i$ 和 $b_i$ 的數字」}
    \only<2-> {
        \\每次都做兩個單點修改
    }

    \only<3> {
        時間複雜度:一次詢問 $O(1)$ 次 multiset 操作,總共 $O((n + q) \log n)$
    }
\end{frame}

\begin{frame}{\btitle{Grades}}
    \begin{problem}[Grades,POI 2017]
        有 $n$ 位學生編號 $1\sim n$ 以任意順序由左到右排成一列,現在你要派給這 $n$ 位學生成績,成績必須是一個介於 $1\sim n$之間的數字,且必須滿足以下條件:

        \begin{itemize}
            \item 若學生 $u$ 的編號比學生 $v$ 大,則學生 $u$ 的成績不可以小於學生 $v$。
            \item 若學生 $v$ 排在學生 $u$ 右邊一位,則學生 $v$ 的成績不可以小於學生 $u$,不然他會很傷心。
        \end{itemize}

            請問最多可以有多少不同的成績種類被派送出去?

            接著會有 $z$ 筆修改,每筆修改會有 $p_i,q_i$,代表交換排在位置 $p_i$ 和 $q_i$ 的學生編號,每次交換後皆須輸出先前問題的答案。
        \begin{itemize}
            \item $1\le n\le 10^6$。
            \item $1\le z\le 3\times10^5$。
        \end{itemize}
    \end{problem}
\end{frame}

\begin{frame}{\btitle{Grades}}
    TL;DR

    有 $n$ 個學生從左到右排成一列,編號大的、排左邊的,成績要比較高。最多能派出幾種不同的成績?

    \only<2-> {
        \only<3>{\emoji{see-no-evil monkey}} \sout<3>{$z$ 次修改,每次交換隊伍裡兩個學生的位置}
    }
\end{frame}

\begin{frame}{\btitle{Grades}}
    $n = 2$......

    \begin{itemize}
        \item $[2, 1]$:兩種
        \item $[1, 2]$:一種
    \end{itemize}
\end{frame}

\begin{frame}{\btitle{Grades}}
    有兩個人 $u < v$......

    \begin{itemize}
        \item $v$ 排 $u$ 左邊\only<2>{:$v$ 的分數本來就該比較高}
        \item $u$ 排 $v$ 左邊\only<2>{:$u, u + 1, \dots, v - 1, v$ 的分數全都要一樣!}
    \end{itemize}
\end{frame}

\begin{frame}{\btitle{Grades}}
    看兩兩相鄰學生的編號關係,獲得(最多)$n - 1$ 條限制「$u_i, u_i + 1, \dots, v_i - 1, v_i$ 的分數要一樣」

    \only<2> {
        要怎麼詢問「最多能派出幾種不同的成績」?
    }

    \only<3> {
        每條限制都是「$u_i + 1, \dots, v_i - 1, v_i$ 的分數都固定了,只有 $u_i$ 的分數可以自由決定」\\
        最後數數看有幾個人的分數可以自由決定
    }

    \only<4> {
        每條限制都是「\yum{$u_i + 1, \dots, v_i - 1, v_i$} 的分數都固定了,只有 $u_i$ 的分數可以自由決定」\\
        最後數數看有幾個人的分數可以自由決定

        \yum{區間加值($\pm 1$),數數看全域有幾個 $0$}
    }
\end{frame}

\begin{frame}{\btitle{Grades}}
    帶修改(交換兩人位置)?
    
    依然可以是兩次單點修改
\end{frame}

\begin{frame}{\btitle{Grades}}
    區間加值、單點修改、數全域有幾個 $0$

    拿你最喜歡的資料結構砸掉 \\
    時間複雜度 $O((n + z) \log n)$
\end{frame}

\begin{frame}{\ebtitle}
    我們是怎麼做完前面兩題的?

    \begin{itemize}
        \item 題目要維護的東西難以維護(整個排列的長相)
        \item 找到小小的特徵點,用小特徵湊出題目要的條件(排列中相鄰元素關係)
        \item 小特徵足夠單純可以維護(multiset、線段樹)
    \end{itemize}
\end{frame}

\begin{frame}{\btitle{Seats}}
    \begin{problem}[Seats,IOI 2018]
        (完整敘述請見講義)\\

        有 $H \times W$ 個位子排成一個矩形,還有 $HW$ 位選手每人分別佔一個位置。有幾個\yum{矩形},使得矩形區域內坐的選手的編號恰好是 $0$ 開始的連續數字?\\

        支援 $Q$ 次修改,每次交換兩位選手的位置,每次交換完輸出以上問題的答案。\\

        \begin{itemize}
            \item $1 \le H \times W \le 10^6$
            \item $1 \le Q \le 5 \times 10^4$
        \end{itemize}
    \end{problem}
\end{frame}

\begin{frame}{\btitle{Seats}}
    \todo 插圖
\end{frame}

\begin{frame}{\btitle{Seats}}
    \begin{problem}[Seats,IOI 2018]
        Subtasks

        \begin{itemize}
            \item (5) $HW \le 100$,$Q \le 5000$
            \item (6) $HW \le 10^4$,$Q \le 5000$
            \item (20) $H \le 1000$,$W \le 1000$,$Q \le 5000$
            \item (6) $Q \le 5000$,對於每次交換 $|a - b| \le 10^4$
            \item (33) $H = 1$
            \item (30) 無額外限制
        \end{itemize}
    \end{problem}
    
    拿零分還可以金牌的難題!?
\end{frame}

\begin{frame}{\btitle{Seats}}
    選手一個一個坐進去,檢查他們是不是坐成矩形的樣子

    躲不開的障礙:\\
    要怎麼檢查一個矩形範圍是不是好的?\\
    要怎麼檢查 $0, \dots, rc - 1$ 的範圍是不是好的?
\end{frame}

\begin{frame}{\btitle{Seats}}
    \begin{problem}[Seats,IOI 2018]
        Subtasks

        \begin{itemize}
            \item (5) $HW \le 100$,$Q \le 5000$
            \item (6) $HW \le 10^4$,$Q \le 5000$
            \item (20) $H \le 1000$,$W \le 1000$,$Q \le 5000$
            \item (6) $Q \le 5000$,對於每次交換 $|a - b| \le 10^4$
            \item \tikzoverlay{nd0}{}\yum{(33) $H = 1$}
            \item (30) 無額外限制
        \end{itemize}
    \end{problem}

    二維太荒謬了,先想辦法搞定一維

    \begin{tikzpicture}[remember picture, overlay]
        \node[draw=WildStrawberry, very thick, left=0.1cm of nd0, anchor=base west] {\phantom{\yum{(33) $H = 1$}}};
    \end{tikzpicture}
\end{frame}

\begin{frame}{\btitle{Seats}}
    選手一個一個坐進去,檢查他們是不是坐成連續區間的樣子

    躲不開的障礙:\\
    要怎麼檢查一個區間是不是好的?\\
    要怎麼檢查 $0, \dots, rc - 1$ 的範圍是不是好的?
\end{frame}

\begin{frame}{\btitle{Seats}}
    \todo
\end{frame}

\begin{frame}{\btitle{Seats}}
    把 $0, \dots, rc - 1$ 塗黑色,其他格子和界外留白。他們剛好在一個連續區間的\yum{充要條件}是......

    \only<2> {
        對於每一組相鄰的格子,\yum{恰好有兩組是一黑一白} \\
    }
\end{frame}

\begin{frame}{\btitle{Seats}}
    \todo
\end{frame}

\begin{frame}{\btitle{Seats}}
    把 $0, 1, \dots, HW - 1$ 一個一個塗黑,每個時間點檢查是不是恰好兩組格子是一黑一白
\end{frame}

\begin{frame}{\btitle{Seats}}
    對於每一組相鄰的格子,他們在\only<1>{哪些時間是一黑一白?}\only<2->{\strong{某個連續的時間區間}是一黑一白}

    \only<2> {
        拿出你最喜歡的資料結構,維護每個時間點一黑一白的格子有幾組,數數看全域有幾個 $2$
    }

    \only<3> {
        拿出你最喜歡的資料結構,維護每個時間點一黑一白的格子有幾組,\yum{數數看全域有幾個 $2$???}
    }

    \only<4> {
        只要有格子是黑的,一黑一白的格子就至少有兩組

        拿出你最喜歡的資料結構,維護每個時間點一黑一白的格子有幾組,\strong{檢查全域最小值是不是 $2$、數數看最小值有幾個}
    }
\end{frame}

\begin{frame}{\btitle{Seats}}
    修改?還是可以兩次單點修改

    時間複雜度:$O((W + Q) \log W)$
\end{frame}

\begin{frame}{\btitle{Seats}}
    \begin{problem}[Seats,IOI 2018]
        Subtasks

        \begin{itemize}
            \item (5) $HW \le 100$,$Q \le 5000$
            \item (6) $HW \le 10^4$,$Q \le 5000$
            \item (20) $H \le 1000$,$W \le 1000$,$Q \le 5000$
            \item (6) $Q \le 5000$,對於每次交換 $|a - b| \le 10^4$
            \item (33) $H = 1$
            \item \tikzoverlay{nd0}{}\yum{(30) 無額外限制}
        \end{itemize}
    \end{problem}

    \begin{tikzpicture}[remember picture, overlay]
        \node[draw=WildStrawberry, very thick, left=0.1cm of nd0, anchor=base west] {\phantom{\yum{(30) 無額外限制}}};
    \end{tikzpicture}
\end{frame}

\begin{frame}{\btitle{Seats}}
    \todo
\end{frame}

\begin{frame}{\btitle{Seats}}
    把 $0, \dots, rc - 1$ 塗黑色,其他格子和界外留白。他們剛好形成一個矩形的\yum{充要條件}是......

    \only<2> {
        對於每一塊 $2 \times 2$ 相鄰的格子,
        \begin{itemize}
            \item 恰好四塊是一黑三白
        \end{itemize}
    }
\end{frame}

\begin{frame}{\btitle{Seats}}
    \todo:甜甜圈
\end{frame}

\begin{frame}{\btitle{Seats}}
    把 $0, \dots, rc - 1$ 塗黑色,其他格子和界外留白。他們剛好形成一個矩形的\yum{充要條件}是......

    對於每一塊 $2 \times 2$ 相鄰的格子,
    \begin{itemize}
        \item 恰好 $4$ 塊是一黑三白
        \item \strong{沒有任何一塊是三黑一白}
    \end{itemize}
\end{frame}

\begin{frame}{\btitle{Seats}}
    \todo
\end{frame}

\begin{frame}{\btitle{Seats}}
    對於每一組 $2 \times 2$ 的格子,他們在\only<1>{哪些時間是一黑三白、或是三黑一白?}\only<2->{\strong{某個連續的時間區間}是一黑三白、或是三黑一白}

    \only<2> {
        只要有格子是黑的,一黑三白的格子就至少有 $4$ 組

        拿出你最喜歡的資料結構,維護每個時間點一黑三白、三黑一白的格子有幾組,\strong{檢查全域最小值是不是 $4$、數數看最小值有幾個}
    }
\end{frame}

\begin{frame}{\btitle{Seats}}
    修改?還是可以兩次單點修改

    \only<1> {
        時間複雜度:$O((HW + Q) \log HW)$ \\
        (多花點力氣 $O(HW + Q \log HW)$)
    }

    \only<2> {
        時間複雜度:$O(HW + \yum{16}Q \log HW)$ \\
        常數巨大!
    }
\end{frame}

\begin{frame}{\btitle{好的連續子序列}}
    \begin{problem}[好的連續子序列,台大演算法設計與分析(ADA)作業]
        給定一個 $1, 2, \dots, N$ 的排列,試求有多少子區間 $[l,r]$,滿足該子區間是一個連續正整數的排列?

        \begin{itemize}
            \item $1\le N \le 5\times10^5$
        \end{itemize}
    \end{problem}

    在 Codeforces 526F 有可以傳的 Judge
\end{frame}

\begin{frame}{\btitle{好的連續子序列}}
    跟一維的 Seats 比起來,少了修改,多了不是 $1$ 開頭的區間也要數數看

    \only<2-> {
        \begin{itemize}
            \item 用 Seats 作法找出 $1$ 開頭的好區間有幾個
            \item 把 $1$ 拿掉(讓他永遠是白色),找出 $2$ 開頭的好區間有幾個
            \item ......
            \item 把 $1, 2, \dots, N - 1$ 拿掉(讓他們永遠是白色),找出 $N$ 開頭的好區間有幾個
        \end{itemize}
    }

    \only<3> {
        題目沒叫你修改,但是你自己把「枚舉排列的開頭」當成 $N$ 次修改
    }
\end{frame}

\begin{frame}{\btitle{好的連續子序列}}
    \only<1> {
        時間複雜度:$O(N \log N)$
    }

    \only<2> {
        時間複雜度:\yum{至少 $O(4N \log N)$}

        常數巨大!我在 NEOJ 吃 TLE
    }
\end{frame}

\begin{frame}{\btitle{好的連續子序列:番外}}
    本題官方作法是分治,也有其他使用大資料結構但可以時限內通過的作法

    你能想到幾種不同的作法?
\end{frame}

\begin{frame}{\btitle{好的連續子序列:番外}}
    關鍵觀察:好區間 $\iff r - l = \max_{l \leq i \leq r}(a_i) - \min_{l \leq i \leq r}(a_i)$

    \only<2> {
        分治作法($O(N \log N)$)
    
        \begin{itemize}
            \item 序列切兩半,數跨兩邊的好的子序列
            \begin{itemize}
                \item 分四種情況:最大值和最小值分別在左半邊還是右半邊
                \item 時間複雜度 $O(N)$
            \end{itemize}
            \item 兩邊遞迴分治
        \end{itemize}
    }
    \only<3> {
        資料結構作法($O(N \log N)$)
    
        \begin{itemize}
            \item 掃描線枚舉區間右界,用資結維護每個左界對應到的 $\left(\left(\max - \min\right) - \left(r - l\right)\right)$
            \begin{itemize}
                \item $\max, \min$ 都可以單調隊列區間加值
                \item 數數看有幾個 $0$
                \item 這坨總是 $\geq 0$,可以數最小值個數
            \end{itemize}
        \end{itemize}
    }
\end{frame}

\begin{frame}{\ebtitle}
    \begin{itemize}
        \item 題目要維護的東西難以維護
        \item 找到小小的特徵點,用小特徵湊出題目要的條件
        \item 小特徵足夠單純可以維護
    \end{itemize}

    資結不是重點,重點是發現精妙的轉換和觀察
\end{frame}
