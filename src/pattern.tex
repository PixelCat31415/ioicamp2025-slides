\section{Pattern}

\begin{frame}{\ebtitle}
    \begin{quote}
        資料結構往往不會赤裸出現 \\
    
        不是「我要用這個資結砸掉這題」\\
        而是「這題需要這樣做,所以可以拿這個資結砸掉」\\
        永遠都先想怎麼做,再尋找合適的資結幫助你 \\

        -- 2024 基礎資結
    \end{quote}

    \only<2> {
        有時候,題目要你維護的東西實在是太荒謬了,你需要自己創造可以維護的東西去維護
    }
\end{frame}

\newcommand{\bigEmoji}[1]{\scalebox{2.5}{\twemoji{#1}}}
\newcommand{\mediumEmoji}[1]{\raisebox{-0.05em}{\scalebox{1.7}{\twemoji{#1}}}}
\begin{frame}{\btitle{Taxis}}
    \only<1,6-> {
        \begin{problem}[Taxis,POI 2018]
            現在有 $n$ 台計程車編號 $1\sim n$,對於第 $i$ 台計程車,給定 $s_i,c_i$,代表該台計程車的收費方式為 $s_i + d\times c_i$,其中 $d$ 為里程數。
    
            給定一個 $1\sim n$的排列,請問是否存在一個里程數 $x/y$,使得把所有計程車的編號照著收費由小到大列出\only<1-7>{恰好是這個排列}\only<8>{\yum{恰好是這個排列???}}(當收費價格一樣,順序可以任意安排),若存在的話請輸出任意一組解,否則輸出「NIE」。
    
            \only<1-6> {
                \only<4>{接著會有 $q$ 筆修改}\only<6>{\yum{接著會有 $q$ 筆修改}},每筆修改會有 $a_i,b_i$,代表交換排列在 $a_i$ 和 $b_i$ 的數字,每次交換後皆須輸出先前問題的答案。\only<6>{\yum{???}}
            }
    
            \begin{itemize}
                \item $1\le n\le 5\times10^5$。
                \item $1\le q\le 5\times10^5$。
            \end{itemize}
        \end{problem}
    }

    \only<2-5> {
        \begin{centikz}[scale=0.9]
            \draw[color=gray, ->] (-5.5, 0) -- (5.5, 0);
            \draw[color=gray, ->] (-5, -0.5) -- (-5, 5.5);
            \node[gray, anchor=north] at (5.5, 0) {\bigEmoji{sport utility vehicle}\bigEmoji{dashing away}};
            \node[gray, anchor=east]  at (-5, 5.5) {\bigEmoji{moneybag}};
            \draw[name path=line1, color=Dandelion, thick] plot[domain=-5.5:5.5] (\x,{-0.3 * \x + 2.1});
            \draw[name path=line2, color=Dandelion, thick] plot[domain=-5.5:5.5] (\x,{0.1 * \x + 3.0});
            \draw[name path=line3, color=Dandelion, thick] plot[domain=-5.5:5.5] (\x,{0.4 * \x + 2.7});
            \node[Dandelion, anchor=west] at (5.6, -0.3 * 5.5 + 2.1 + 0.1) {\underline{\mediumEmoji{taxi} $1$}};
            \node[Dandelion, anchor=west] at (5.6,  0.1 * 5.5 + 3.0 + 0.1) {\underline{\mediumEmoji{taxi} $2$}};
            \node[Dandelion, anchor=west] at (5.6,  0.4 * 5.5 + 2.7 + 0.1) {\underline{\mediumEmoji{taxi} $3$}};
            \only<3> {
                \draw[name path=vline, color=blue, dashed, very thick] (-1.5, -0.5) -- (-1.5, 5.5);
                \node[anchor=south] at (-1.5, 5.5) {$[3, 1, 2]$ \emoji{check mark button}};
                \fill [blue, right, name intersections={of=vline and line1}] (intersection-1) circle (2pt) node {$1$};
                \fill [blue, above right, name intersections={of=vline and line2}] (intersection-1) circle (2pt) node {$2$};
                \fill [blue, below right, name intersections={of=vline and line3}] (intersection-1) circle (2pt) node {$3$};
            }
            \only<4> {
                \draw[name path=vline, color=blue, dashed, very thick] (0, -0.5) -- (0, 5.5);
                \node[anchor=south] at (0, 5.5) {$[1, 3, 2]$ \emoji{check mark button}};
                \fill [blue, right, name intersections={of=vline and line1}] (intersection-1) circle (2pt) node {$1$};
                \fill [blue, above right, name intersections={of=vline and line2}] (intersection-1) circle (2pt) node {$2$};
                \fill [blue, right, name intersections={of=vline and line3}] (intersection-1) circle (2pt) node {$3$};
            }
            \only<5> {
                \node[anchor=south] at (0, 5.5) {$[2, 3, 1]$ \emoji{cross mark}};
            }
        \end{centikz}
    }
\end{frame}

\begin{frame}{\btitle{Taxis}}
    什麼時候......

    \begin{itemize}
        \item \only<2->{\emoji{vomiting_face}} \sout<2->{計程車照收費排序剛好是 $1, 2, \dots, n$}
        \only<3-> {
            \item \only<4>{\emoji{smiling face with hearts}} $1$ 號車收費 $\leq 2$ 號,而且
            \item \only<4>{\emoji{smiling face with hearts}} $2$ 號車收費 $\leq 3$ 號,而且
            \item ......
            \item \only<4>{\emoji{smiling face with hearts}} $n - 1$ 號車收費 $\leq n$ 號
        }
    \end{itemize}

    \only<4> {
        $n = 2$ 我們總會做了吧?
    }
\end{frame}

\begin{frame}{\btitle{Taxis}}
    只考慮 $i$ 號車和 $i + 1$ 號車,可以滿足「$i$ 在 $i + 1$ 前面」的距離 $d$ 是一個 $d \leq \square$ 或 $d \geq \square$ 的限制

    當每一組相鄰計程車的順序都滿足,所有車就會照順序排好
    
    \only<2> {
        維護一坨射線有沒有交集 \\
        可以用兩個 multiset 維護
    }
\end{frame}

\begin{frame}{\btitle{Taxis}}
    帶修改?

    \sout<2->{「每筆修改會有 $a_i,b_i$,代表交換排列在 $a_i$ 和 $b_i$ 的數字」}
    \only<2-> {
        \\每次都做兩個單點修改
    }

    \only<3> {
        時間複雜度:一次詢問 $O(1)$ 次 multiset 操作,總共 $O((n + q) \log n)$
    }
\end{frame}

\begin{frame}{\btitle{Grades}}
    \begin{problem}[Grades,POI 2017]
        有 $n$ 位學生編號 $1\sim n$ 以任意順序由左到右排成一列,現在你要派給這 $n$ 位學生成績,成績必須是一個介於 $1\sim n$之間的數字,且必須滿足以下條件:

        \begin{itemize}
            \item 若學生 $u$ 的編號比學生 $v$ 大,則學生 $u$ 的成績不可以小於學生 $v$。
            \item 若學生 $v$ 排在學生 $u$ 右邊一位,則學生 $v$ 的成績不可以小於學生 $u$,不然他會很傷心。
        \end{itemize}

            請問最多可以有多少不同的成績種類被派送出去?

            接著會有 $z$ 筆修改,每筆修改會有 $p_i,q_i$,代表交換排在位置 $p_i$ 和 $q_i$ 的學生編號,每次交換後皆須輸出先前問題的答案。
        \begin{itemize}
            \item $1\le n\le 10^6$。
            \item $1\le z\le 3\times10^5$。
        \end{itemize}
    \end{problem}
\end{frame}

\begin{frame}{\btitle{Grades}}
    TL;DR

    有 $n$ 個學生從左到右排成一列,編號大的、排左邊的,成績要比較高。最多能派出幾種不同的成績?

    \only<2-> {
        \only<3>{\emoji{see-no-evil monkey}} \sout<3>{$z$ 次修改,每次交換隊伍裡兩個學生的位置}
    }
\end{frame}

\begin{frame}{\btitle{Grades}}
    $n = 2$......

    \begin{itemize}
        \item $[2, 1]$:兩種
        \item $[1, 2]$:一種
    \end{itemize}
\end{frame}

\begin{frame}{\btitle{Grades}}
    有兩個人 $u < v$......

    \begin{itemize}
        \item $v$ 排 $u$ 左邊\only<2>{:$v$ 的分數本來就該比較高}
        \item $u$ 排 $v$ 左邊\only<2>{:$u, u + 1, \dots, v - 1, v$ 的分數全都要一樣!}
    \end{itemize}
\end{frame}

\begin{frame}{\btitle{Grades}}
    看兩兩相鄰學生的編號關係,獲得(最多)$n - 1$ 條限制「$u_i, u_i + 1, \dots, v_i - 1, v_i$ 的分數要一樣」

    \only<2> {
        要怎麼詢問「最多能派出幾種不同的成績」?
    }

    \only<3> {
        每條限制都是「$u_i + 1, \dots, v_i - 1, v_i$ 的分數都固定了,只有 $u_i$ 的分數可以自由決定」\\
        最後數數看有幾個人的分數可以自由決定
    }

    \only<4> {
        每條限制都是「\yum{$u_i + 1, \dots, v_i - 1, v_i$} 的分數都固定了,只有 $u_i$ 的分數可以自由決定」\\
        最後數數看有幾個人的分數可以自由決定

        \yum{區間加值($\pm 1$),數數看全域有幾個 $0$}
    }
\end{frame}

\begin{frame}{\btitle{Grades}}
    帶修改(交換兩人位置)?
    
    依然可以是兩次單點修改
\end{frame}

\begin{frame}{\btitle{Grades}}
    區間加值、單點修改、數全域有幾個 $0$

    拿你最喜歡的資料結構砸掉 \\
    時間複雜度 $O((n + z) \log n)$
\end{frame}

\begin{frame}{\ebtitle}
    我們是怎麼做完前面兩題的?

    \begin{itemize}
        \item 題目要維護的東西難以維護(整個排列的長相)
        \item 找到小小的特徵點,用小特徵湊出題目要的條件(排列中相鄰元素關係)
        \item 小特徵足夠單純可以維護(multiset、線段樹)
    \end{itemize}
\end{frame}

\begin{frame}{\btitle{Seats}}
    \begin{problem}[Seats,IOI 2018]
        (完整敘述請見講義)\\

        有 $H \times W$ 個位子排成一個矩形,還有 $HW$ 位選手每人分別佔一個位置。有幾個\yum{矩形},使得矩形區域內坐的選手的編號恰好是 $0$ 開始的連續數字?\\

        支援 $Q$ 次修改,每次交換兩位選手的位置,每次交換完輸出以上問題的答案。\\

        \begin{itemize}
            \item $1 \le H \times W \le 10^6$
            \item $1 \le Q \le 5 \times 10^4$
        \end{itemize}
    \end{problem}
\end{frame}

\newcommand{\DrawSeatsA}{
    \draw[help lines, xshift=-0.5cm, yshift=-0.5cm] (0, 0) grid (3, 2);
    \node at (0, 1) {4}; \node at (1, 1) {0}; \node at (2, 1) {3};
    \node at (0, 0) {5}; \node at (1, 0) {1}; \node at (2, 0) {2};
}
\newcommand{\DrawSeatsB}{
    \draw[help lines, xshift=-0.5cm, yshift=-0.5cm] (0, 0) grid (3, 2);
    \node at (0, 1) {4}; \node at (1, 1) {5}; \node at (2, 1) {3};
    \node at (0, 0) {0}; \node at (1, 0) {1}; \node at (2, 0) {2};
}
\newcommand{\DrawSeatsC}{
    \draw[help lines, xshift=-0.5cm, yshift=-0.5cm] (0, 0) grid (3, 2);
    \node at (0, 1) {4}; \node at (1, 1) {5}; \node at (2, 1) {3};
    \node at (0, 0) {0}; \node at (1, 0) {2}; \node at (2, 0) {1};
}

\begin{frame}{\btitle{Seats}}
    \only<1> {
        \begin{centikz}
            \DrawSeatsA

            \tikzset{shift={(4  ,  1.1)}}
            \fill[ForestGreen!10!white] (0.5, 0.5) rectangle (1.5, 1.5);
            \DrawSeatsA
            \draw[ForestGreen, very thick] (0.5, 0.5) rectangle (1.5, 1.5);
            
            \tikzset{shift={(0  , -2.2)}}
            \fill[ForestGreen!10!white] (0.5, -0.5) rectangle (1.5, 1.5);
            \DrawSeatsA
            \draw[ForestGreen, very thick] (0.5, -0.5) rectangle (1.5, 1.5);

            \tikzset{shift={(3.2,  2.2)}}
            \fill[ForestGreen!10!white] (0.5, -0.5) rectangle (2.5, 1.5);
            \DrawSeatsA
            \draw[ForestGreen, very thick] (0.5, -0.5) rectangle (2.5, 1.5);

            \tikzset{shift={(0  , -2.2)}}
            \fill[ForestGreen!10!white] (-0.5, -0.5) rectangle (2.5, 1.5);
            \DrawSeatsA
            \draw[ForestGreen, very thick] (-0.5, -0.5) rectangle (2.5, 1.5);

        \end{centikz}
    }
    \only<2> {
        \begin{centikz}
            \draw[help lines, xshift=-0.5cm, yshift=-0.5cm] (0, 0) grid (3, 2);
            \node at (0, 1) {4}; \node at (1, 1) (nd2) {\yum{5}}; \node at (2, 1) {3};
            \node at (0, 0) (nd1) {\yum{0}}; \node at (1, 0) {1}; \node at (2, 0) {2};
            \draw[red, thick, <->, >={Latex}] (nd1) to (nd2);

            \tikzset{shift={(4  ,  1.1)}}
            \fill[ForestGreen!10!white] (-0.5, -0.5) rectangle (0.5, 0.5);
            \DrawSeatsB
            \draw[ForestGreen, very thick] (-0.5, -0.5) rectangle (0.5, 0.5);
            
            \tikzset{shift={(0  , -2.2)}}
            \fill[ForestGreen!10!white] (-0.5, -0.5) rectangle (1.5, 0.5);
            \DrawSeatsB
            \draw[ForestGreen, very thick] (-0.5, -0.5) rectangle (1.5, 0.5);

            \tikzset{shift={(3.2,  2.2)}}
            \fill[ForestGreen!10!white] (-0.5, -0.5) rectangle (2.5, 0.5);
            \DrawSeatsB
            \draw[ForestGreen, very thick] (-0.5, -0.5) rectangle (2.5, 0.5);

            \tikzset{shift={(0  , -2.2)}}
            \fill[ForestGreen!10!white] (-0.5, -0.5) rectangle (2.5, 1.5);
            \DrawSeatsB
            \draw[ForestGreen, very thick] (-0.5, -0.5) rectangle (2.5, 1.5);

        \end{centikz}
    }
    \only<3> {
        \begin{centikz}
            \draw[help lines, xshift=-0.5cm, yshift=-0.5cm] (0, 0) grid (3, 2);
            \node at (0, 1) {4}; \node at (1, 1) {5}; \node at (2, 1) {3};
            \node at (0, 0) {0}; \node at (1, 0) (nd1) {\yum{2}}; \node at (2, 0) (nd2) {\yum{1}};
            \draw[red, thick, <->, >={Latex}] (nd1) to (nd2);

            \tikzset{shift={(4  ,  1.1)}}
            \fill[ForestGreen!10!white] (-0.5, -0.5) rectangle (0.5, 0.5);
            \DrawSeatsC
            \draw[ForestGreen, very thick] (-0.5, -0.5) rectangle (0.5, 0.5);
            
            \tikzset{shift={(0  , -2.2)}}
            \fill[ForestGreen!10!white] (-0.5, -0.5) rectangle (2.5, 1.5);
            \DrawSeatsC
            \draw[ForestGreen, very thick] (-0.5, -0.5) rectangle (2.5, 1.5);

            \tikzset{shift={(3.2,  2.2)}}
            \fill[ForestGreen!10!white] (-0.5, -0.5) rectangle (2.5, 0.5);
            \DrawSeatsC
            \draw[ForestGreen, very thick] (-0.5, -0.5) rectangle (2.5, 0.5);

        \end{centikz}
    }
\end{frame}

\begin{frame}{\btitle{Seats}}
    \begin{problem}[Seats,IOI 2018]
        Subtasks

        \begin{itemize}
            \item (5) $HW \le 100$,$Q \le 5000$
            \item (6) $HW \le 10^4$,$Q \le 5000$
            \item (20) $H \le 1000$,$W \le 1000$,$Q \le 5000$
            \item (6) $Q \le 5000$,對於每次交換 $|a - b| \le 10^4$
            \item (33) $H = 1$
            \item (30) 無額外限制
        \end{itemize}
    \end{problem}
    
    拿零分還可以金牌的難題!?
\end{frame}

\begin{frame}{\btitle{Seats}}
    選手一個一個坐進去,檢查他們是不是坐成矩形的樣子

    躲不開的障礙:要怎麼檢查 $0, \dots, rc - 1$ 的範圍是不是好的?
\end{frame}

\newcommand{\DrawFadedSeats}[4]{
    \only<#1> { \begin{scope}[opacity=0,transparency group] #4 \end{scope} }
    \only<#2> { #4 }
    \only<#3> { \begin{scope}[opacity=0.4,transparency group] #4 \end{scope} }
}
\begin{frame}{\btitle{Seats}}
    \begin{centikz}
        \only<1> {
            \draw[help lines, xshift=-0.5cm, yshift=-0.5cm] (0, 0) grid (3, 2);
            \node at (0, 1) { }; \node at (1, 1) { }; \node at (2, 1) { };
            \node at (0, 0) { }; \node at (1, 0) { }; \node at (2, 0) { };
        }

        \DrawFadedSeats{1}{2,8}{3-7}{
            \fill[ForestGreen!20!white] (-0.5, -0.5) rectangle (0.5, 0.5);
            \draw[help lines, xshift=-0.5cm, yshift=-0.5cm] (0, 0) grid (3, 2);
            \node at (0, 1) { }; \node at (1, 1) { }; \node at (2, 1) { };
            \node at (0, 0) {0}; \node at (1, 0) { }; \node at (2, 0) { };
            \draw[ForestGreen, very thick] (-0.5, -0.5) rectangle (0.5, 0.5);
        }
        
        \tikzset{shift={(3.2, 0)}}
        \DrawFadedSeats{1-2}{3,8}{4-7}{
            \fill[ForestGreen!20!white] (-0.5, -0.5) rectangle (1.5, 0.5);
            \draw[help lines, xshift=-0.5cm, yshift=-0.5cm] (0, 0) grid (3, 2);
            \node at (0, 1) { }; \node at (1, 1) { }; \node at (2, 1) { };
            \node at (0, 0) {0}; \node at (1, 0) {1}; \node at (2, 0) { };
            \draw[ForestGreen, very thick] (-0.5, -0.5) rectangle (1.5, 0.5);
        }

        \tikzset{shift={(3.2, 0)}}
        \DrawFadedSeats{1-3}{4,8}{5-7}{
            \fill[ForestGreen!20!white] (-0.5, -0.5) rectangle (2.5, 0.5);
            \draw[help lines, xshift=-0.5cm, yshift=-0.5cm] (0, 0) grid (3, 2);
            \node at (0, 1) { }; \node at (1, 1) { }; \node at (2, 1) { };
            \node at (0, 0) {0}; \node at (1, 0) {1}; \node at (2, 0) {2};
            \draw[ForestGreen, very thick] (-0.5, -0.5) rectangle (2.5, 0.5);
        }

        \tikzset{shift={(-6.4, -2.2)}}
        \DrawFadedSeats{1-4}{5,8}{6-7}{
            \fill[red!10!white] (-0.5, -0.5) rectangle (2.5, 0.5);
            \fill[red!10!white] (1.5, 0.5) rectangle (2.5, 1.5);
            \draw[help lines, xshift=-0.5cm, yshift=-0.5cm] (0, 0) grid (3, 2);
            \node at (0, 1) { }; \node at (1, 1) { }; \node at (2, 1) {3};
            \node at (0, 0) {0}; \node at (1, 0) {1}; \node at (2, 0) {2};
            \draw[red, very thick] (-0.5, -0.5) -- (2.5, -0.5) -- (2.5, 1.5) -- (1.5, 1.5) -- (1.5, 0.5) -- (-0.5, 0.5) -- cycle;
        }

        \tikzset{shift={(3.2, 0)}}
        \DrawFadedSeats{1-5}{6,8}{7}{
            \fill[red!10!white] (-0.5, -0.5) rectangle (2.5, 0.5);
            \fill[red!10!white] (1.5, 0.5) rectangle (2.5, 1.5);
            \fill[red!10!white] (-0.5, 0.5) rectangle (0.5, 1.5);
            \draw[help lines, xshift=-0.5cm, yshift=-0.5cm] (0, 0) grid (3, 2);
            \node at (0, 1) {4}; \node at (1, 1) { }; \node at (2, 1) {3};
            \node at (0, 0) {0}; \node at (1, 0) {1}; \node at (2, 0) {2};
            \draw[red, very thick] (-0.5, -0.5) -- (2.5, -0.5) -- (2.5, 1.5) -- (1.5, 1.5) -- (1.5, 0.5) -- (0.5, 0.5) -- (0.5, 1.5) -- (-0.5, 1.5) -- cycle;
        }

        \tikzset{shift={(3.2, 0)}}
        \only<7-> {
            \fill[ForestGreen!20!white] (-0.5, -0.5) rectangle (2.5, 1.5);
            \draw[help lines, xshift=-0.5cm, yshift=-0.5cm] (0, 0) grid (3, 2);
            \node at (0, 1) {4}; \node at (1, 1) {5}; \node at (2, 1) {3};
            \node at (0, 0) {0}; \node at (1, 0) {1}; \node at (2, 0) {2};
            \draw[ForestGreen, very thick] (-0.5, -0.5) rectangle (2.5, 1.5);
        }
    \end{centikz}
\end{frame}

\begin{frame}{\btitle{Seats}}
    \begin{problem}[Seats,IOI 2018]
        Subtasks

        \begin{itemize}
            \item (5) $HW \le 100$,$Q \le 5000$
            \item (6) $HW \le 10^4$,$Q \le 5000$
            \item (20) $H \le 1000$,$W \le 1000$,$Q \le 5000$
            \item (6) $Q \le 5000$,對於每次交換 $|a - b| \le 10^4$
            \item \tikzoverlay{nd0}{}\yum{(33) $H = 1$}
            \item (30) 無額外限制
        \end{itemize}
    \end{problem}

    二維太荒謬了,先想辦法搞定一維

    \begin{tikzpicture}[remember picture, overlay]
        \node[draw=WildStrawberry, very thick, left=0.1cm of nd0, anchor=base west] {\phantom{\yum{(33) $H = 1$}}};
    \end{tikzpicture}
\end{frame}

\begin{frame}{\btitle{Seats}}
    選手一個一個坐進去,檢查他們是不是坐成連續區間的樣子

    躲不開的障礙:要怎麼檢查 $0, \dots, rc - 1$ 的範圍是不是好的?
\end{frame}

\begin{frame}{\btitle{Seats}}
    \begin{centikz}[
        nddark/.style={rectangle, minimum width=1cm, minimum height=1cm, white, fill=black!80!white, font={\Large}}
    ]
        \draw[help lines, xshift=-0.5cm, yshift=-0.5cm] (0, 0) grid (3, 1);
        \node[nddark] at (0, 0) {1};
        \uncover<2> { \node[anchor=west] at (3, 0) {\scalebox{2}{\emoji{check mark button}}}; }

        \tikzset{shift={(0, -2)}}
        \draw[help lines, xshift=-0.5cm, yshift=-0.5cm] (0, 0) grid (3, 1);
        \node[nddark] at (0, 0) {1};
        \node[nddark] at (2, 0) {2};
        \uncover<2> { \node[anchor=west] at (3, 0) {\scalebox{2}{\emoji{cross mark}}}; }

        \tikzset{shift={(0, -2)}}
        \draw[help lines, xshift=-0.5cm, yshift=-0.5cm] (0, 0) grid (3, 1);
        \node[nddark] at (0, 0) {1};
        \node[nddark] at (1, 0) {3};
        \node[nddark] at (2, 0) {2};
        \uncover<2> { \node[anchor=west] at (3, 0) {\scalebox{2}{\emoji{check mark button}}}; }
    \end{centikz}
\end{frame}

\begin{frame}{\btitle{Seats}}
    把 $0, \dots, rc - 1$ 塗黑色,其他格子和界外留白。他們剛好在一個連續區間的\yum{充要條件}是......

    \only<2> {
        對於每一組相鄰的格子,\yum{恰好有兩組是一黑一白} \\
    }
\end{frame}

\newcommand{\DrawGridA}[1]{
    \draw[help lines, xshift=-0.5cm, yshift=-0.5cm] (0, 0) grid (3, 1);
    \draw[help lines, dashed]
        (-0.5, -0.5) -- (-1.5, -0.5) -- (-1.5, 0.5) -- (-0.5, 0.5)
        (2.5, -0.5) -- (3.5, -0.5) -- (3.5, 0.5) -- (2.5, 0.5);
    \foreach \i in {0, 1, 2, 3, 4} {
        \node[ndempty] at (\i - 1, 0) (nd#1-\i) {};
    }
}
\newcommand{\StarredPair}[3]{
    \draw[Dandelion, very thick]
        (nd#1-#2.north) to[bend left=60] node[fill=white, inner sep=1pt]{\emoji{star}} (nd#1-#3.north);
}
\begin{frame}{\btitle{Seats}}
    \begin{centikz}[
        nddark/.style={rectangle, minimum width=1cm, minimum height=1cm, white, fill=black!80!white, font={\Large}},
        ndempty/.style={rectangle, minimum width=1cm, minimum height=1cm},
    ]
        \DrawGridA{1}
        \node[nddark] at (0, 0) {1};
        \StarredPair{1}{0}{1}
        \StarredPair{1}{1}{2}
        \uncover<2> { \node[anchor=west] at (4, 0) {\scalebox{2}{\emoji{check mark button}}}; }

        \tikzset{shift={(0, -2)}}
        \DrawGridA{2}
        \node[nddark] at (0, 0) {1};
        \node[nddark] at (2, 0) {2};
        \StarredPair{2}{0}{1}
        \StarredPair{2}{1}{2}
        \StarredPair{2}{2}{3}
        \StarredPair{2}{3}{4}
        \uncover<2> { \node[anchor=west] at (4, 0) {\scalebox{2}{\emoji{cross mark}}}; }

        \tikzset{shift={(0, -2)}}
        \DrawGridA{3}
        \node[nddark] at (0, 0) {1};
        \node[nddark] at (1, 0) {3};
        \node[nddark] at (2, 0) {2};
        \StarredPair{3}{0}{1}
        \StarredPair{3}{3}{4}
        \uncover<2> { \node[anchor=west] at (4, 0) {\scalebox{2}{\emoji{check mark button}}}; }
    \end{centikz}
\end{frame}

\begin{frame}{\btitle{Seats}}
    把 $0, 1, \dots, HW - 1$ 一個一個塗黑,每個時間點檢查是不是恰好兩組格子是一黑一白
\end{frame}

\begin{frame}{\btitle{Seats}}
    對於每一組相鄰的格子,他們在\only<1>{哪些時間是一黑一白?}\only<2->{\strong{某個連續的時間區間}是一黑一白}

    \only<2> {
        拿出你最喜歡的資料結構,維護每個時間點一黑一白的格子有幾組,數數看全域有幾個 $2$
    }

    \only<3> {
        拿出你最喜歡的資料結構,維護每個時間點一黑一白的格子有幾組,\yum{數數看全域有幾個 $2$???}
    }

    \only<4> {
        只要有格子是黑的,一黑一白的格子就至少有兩組

        拿出你最喜歡的資料結構,維護每個時間點一黑一白的格子有幾組,\strong{檢查全域最小值是不是 $2$、數數看最小值有幾個}
    }
\end{frame}

\begin{frame}{\btitle{Seats}}
    修改?還是可以兩次單點修改

    時間複雜度:$O((W + Q) \log W)$
\end{frame}

\begin{frame}{\btitle{Seats}}
    \begin{problem}[Seats,IOI 2018]
        Subtasks

        \begin{itemize}
            \item (5) $HW \le 100$,$Q \le 5000$
            \item (6) $HW \le 10^4$,$Q \le 5000$
            \item (20) $H \le 1000$,$W \le 1000$,$Q \le 5000$
            \item (6) $Q \le 5000$,對於每次交換 $|a - b| \le 10^4$
            \item (33) $H = 1$
            \item \tikzoverlay{nd0}{}\yum{(30) 無額外限制}
        \end{itemize}
    \end{problem}

    \begin{tikzpicture}[remember picture, overlay]
        \node[draw=WildStrawberry, very thick, left=0.1cm of nd0, anchor=base west] {\phantom{\yum{(30) 無額外限制}}};
    \end{tikzpicture}
\end{frame}

\begin{frame}{\btitle{Seats}}
    \begin{centikz}[
        nddark/.style={rectangle, minimum width=1cm, minimum height=1cm, white, fill=black!80!white, font={\Large}}
    ]
        \draw[help lines, xshift=-0.5cm, yshift=-0.5cm] (0, 0) grid (2, 2);
        \node[nddark] at (0, 0) {};
        \node[nddark] at (0, 1) {};
        \node[nddark] at (1, 1) {};
        \uncover<2> { \node[anchor=west] at (2, 0.5) {\scalebox{2}{\emoji{cross mark}}}; }

        \tikzset{shift={(0, -3.5)}}
        \draw[help lines, xshift=-0.5cm, yshift=-0.5cm] (0, 0) grid (2, 2);
        \node[nddark] at (1, 0) {};
        \node[nddark] at (0, 1) {};
        \uncover<2> { \node[anchor=west] at (2, 0.5) {\scalebox{2}{\emoji{cross mark}}}; }

        \tikzset{shift={(5, 3.5)}}
        \draw[help lines, xshift=-0.5cm, yshift=-0.5cm] (0, 0) grid (2, 2);
        \node[nddark] at (0, 1) {};
        \node[nddark] at (1, 1) {};
        \uncover<2> { \node[anchor=west] at (2, 0.5) {\scalebox{2}{\emoji{check mark button}}}; }

        \tikzset{shift={(0, -3.5)}}
        \draw[help lines, xshift=-0.5cm, yshift=-0.5cm] (0, 0) grid (2, 2);
        \node[nddark] at (0, 0) {};
        \node[nddark] at (0, 1) {};
        \node[nddark] at (1, 0) {};
        \node[nddark] at (1, 1) {};
        \uncover<2> { \node[anchor=west] at (2, 0.5) {\scalebox{2}{\emoji{check mark button}}}; }
    \end{centikz}
\end{frame}

\begin{frame}{\btitle{Seats}}
    把 $0, \dots, rc - 1$ 塗黑色,其他格子和界外留白。他們剛好形成一個矩形的\yum{充要條件}是......

    \only<2> {
        對於每一塊 $2 \times 2$ 相鄰的格子,
        \begin{itemize}
            \item 恰好四塊是一黑三白
        \end{itemize}
    }
\end{frame}

\newcommand{\StarredBlock}[2]{
    \node[scale=0.7, Dandelion, circle, draw, very thick, fill=Gold!30!white, inner sep=0]
        at (#1 - 0.5, #2 - 0.5) {\emoji{star}};
    % \node[draw=Dandelion, very thick, circle, inner sep=0, fill=white, fill opacity=0.7]
    %     at (#1 - 0.5, #2 - 0.5) {\emoji{star}};
    % \fill[Dandelion, very thick] (#1 - 0.5, #2 - 0.5) circle (4pt);
}
\newcommand{\FiredBlock}[2]{
    \node[scale=0.7, Crimson, circle, draw, very thick, fill=Crimson!30!white, inner sep=0]
        at (#1 - 0.5, #2 - 0.5) {\emoji{fire}};
}
\begin{frame}{\btitle{Seats}}
    \begin{centikz}[
        nddark/.style={rectangle, minimum width=1cm, minimum height=1cm, white, fill=black!80!white, font={\Large}}
    ]
        \draw[help lines, xshift=-0.5cm, yshift=-0.5cm] (0, 0) grid (2, 2);
        \node[nddark] at (0, 0) {};
        \node[nddark] at (0, 1) {};
        \node[nddark] at (1, 1) {};
        \StarredBlock{0}{0}
        \StarredBlock{1}{0}
        \StarredBlock{2}{1}
        \StarredBlock{0}{2}
        \StarredBlock{2}{2}
        \node[anchor=west] at (2, 0.5) {\scalebox{2}{\emoji{cross mark}}};

        \tikzset{shift={(0, -3.5)}}
        \draw[help lines, xshift=-0.5cm, yshift=-0.5cm] (0, 0) grid (2, 2);
        \node[nddark] at (1, 0) {};
        \node[nddark] at (0, 1) {};
        \StarredBlock{0}{1}
        \StarredBlock{0}{2}
        \StarredBlock{1}{2}
        \StarredBlock{1}{0}
        \StarredBlock{2}{0}
        \StarredBlock{2}{1}
        \node[anchor=west] at (2, 0.5) {\scalebox{2}{\emoji{cross mark}}};

        \tikzset{shift={(5, 3.5)}}
        \draw[help lines, xshift=-0.5cm, yshift=-0.5cm] (0, 0) grid (2, 2);
        \node[nddark] at (0, 1) {};
        \node[nddark] at (1, 1) {};
        \StarredBlock{0}{1}
        \StarredBlock{0}{2}
        \StarredBlock{2}{1}
        \StarredBlock{2}{2}
        \node[anchor=west] at (2, 0.5) {\scalebox{2}{\emoji{check mark button}}};

        \tikzset{shift={(0, -3.5)}}
        \draw[help lines, xshift=-0.5cm, yshift=-0.5cm] (0, 0) grid (2, 2);
        \node[nddark] at (0, 0) {};
        \node[nddark] at (0, 1) {};
        \node[nddark] at (1, 0) {};
        \node[nddark] at (1, 1) {};
        \StarredBlock{0}{0}
        \StarredBlock{0}{2}
        \StarredBlock{2}{0}
        \StarredBlock{2}{2}
        \node[anchor=west] at (2, 0.5) {\scalebox{2}{\emoji{check mark button}}};
    \end{centikz}
\end{frame}

\begin{frame}{\btitle{Seats}}
    \begin{centikz}[
        scale=0.9,
        nddark/.style={rectangle, minimum width=0.9cm, minimum height=0.9cm, white, fill=black!80!white, font={\Large}}
    ]
        \draw[help lines, xshift=-0.5cm, yshift=-0.5cm] (0, 0) grid (4, 4);
        \node[nddark] at (0, 0) {};
        \node[nddark] at (0, 1) {};
        \node[nddark] at (0, 2) {};
        \node[nddark] at (0, 3) {};
        \node[nddark] at (1, 0) {};
        \node[nddark] at (1, 1) {};
        \node[nddark] at (1, 3) {};
        \node[nddark] at (2, 0) {};
        \node[nddark] at (2, 1) {};
        \node[nddark] at (2, 3) {};
        \node[nddark] at (3, 0) {};
        \node[nddark] at (3, 1) {};
        \node[nddark] at (3, 2) {};
        \node[nddark] at (3, 3) {};
        \StarredBlock{0}{0}
        \StarredBlock{4}{0}
        \StarredBlock{0}{4}
        \StarredBlock{4}{4}

        \tikzset{shift={(6, 0)}}
        \draw[help lines, xshift=-0.5cm, yshift=-0.5cm] (0, 0) grid (4, 4);
        \node[nddark] at (0, 0) {};
        \node[nddark] at (0, 1) {};
        \node[nddark] at (0, 2) {};
        \node[nddark] at (0, 3) {};
        \node[nddark] at (1, 0) {};
        \node[nddark] at (1, 1) {};
        \node[nddark] at (1, 3) {};
        \node[nddark] at (2, 0) {};
        \node[nddark] at (2, 2) {};
        \node[nddark] at (2, 3) {};
        \node[nddark] at (3, 0) {};
        \node[nddark] at (3, 1) {};
        \node[nddark] at (3, 2) {};
        \node[nddark] at (3, 3) {};
        \StarredBlock{0}{0}
        \StarredBlock{4}{0}
        \StarredBlock{0}{4}
        \StarredBlock{4}{4}
    \end{centikz}
\end{frame}

\begin{frame}{\btitle{Seats}}
    把 $0, \dots, rc - 1$ 塗黑色,其他格子和界外留白。他們剛好形成一個矩形的\yum{充要條件}是......

    對於每一塊 $2 \times 2$ 相鄰的格子,
    \begin{itemize}
        \item 恰好 $4$ 塊是一黑三白
        \item \strong{沒有任何一塊是三黑一白}
    \end{itemize}
\end{frame}

\begin{frame}{\btitle{Seats}}
    \begin{centikz}[
        scale=0.9,
        nddark/.style={rectangle, minimum width=0.9cm, minimum height=0.9cm, white, fill=black!80!white, font={\Large}}
    ]
        \draw[help lines, xshift=-0.5cm, yshift=-0.5cm] (0, 0) grid (4, 4);
        \node[nddark] at (0, 0) {};
        \node[nddark] at (0, 1) {};
        \node[nddark] at (0, 2) {};
        \node[nddark] at (0, 3) {};
        \node[nddark] at (1, 0) {};
        \node[nddark] at (1, 1) {};
        \node[nddark] at (1, 3) {};
        \node[nddark] at (2, 0) {};
        \node[nddark] at (2, 1) {};
        \node[nddark] at (2, 3) {};
        \node[nddark] at (3, 0) {};
        \node[nddark] at (3, 1) {};
        \node[nddark] at (3, 2) {};
        \node[nddark] at (3, 3) {};
        \StarredBlock{0}{0}
        \StarredBlock{4}{0}
        \StarredBlock{0}{4}
        \StarredBlock{4}{4}
        \FiredBlock{1}{2}
        \FiredBlock{1}{3}
        \FiredBlock{3}{2}
        \FiredBlock{3}{3}

        \tikzset{shift={(6, 0)}}
        \draw[help lines, xshift=-0.5cm, yshift=-0.5cm] (0, 0) grid (4, 4);
        \node[nddark] at (0, 0) {};
        \node[nddark] at (0, 1) {};
        \node[nddark] at (0, 2) {};
        \node[nddark] at (0, 3) {};
        \node[nddark] at (1, 0) {};
        \node[nddark] at (1, 1) {};
        \node[nddark] at (1, 3) {};
        \node[nddark] at (2, 0) {};
        \node[nddark] at (2, 2) {};
        \node[nddark] at (2, 3) {};
        \node[nddark] at (3, 0) {};
        \node[nddark] at (3, 1) {};
        \node[nddark] at (3, 2) {};
        \node[nddark] at (3, 3) {};
        \StarredBlock{0}{0}
        \StarredBlock{4}{0}
        \StarredBlock{0}{4}
        \StarredBlock{4}{4}
        \FiredBlock{1}{2}
        \FiredBlock{1}{3}
        \FiredBlock{2}{3}
        \FiredBlock{2}{1}
        \FiredBlock{3}{1}
        \FiredBlock{3}{2}
    \end{centikz}
\end{frame}

\begin{frame}{\btitle{Seats}}
    對於每一組 $2 \times 2$ 的格子,他們在\only<1>{哪些時間是一黑三白、或是三黑一白?}\only<2->{\strong{某個連續的時間區間}是一黑三白、或是三黑一白}

    \only<2> {
        只要有格子是黑的,一黑三白的格子就至少有 $4$ 組

        拿出你最喜歡的資料結構,維護每個時間點一黑三白、三黑一白的格子有幾組,\strong{檢查全域最小值是不是 $4$、數數看最小值有幾個}
    }
\end{frame}

\begin{frame}{\btitle{Seats}}
    修改?還是可以兩次單點修改

    \only<1> {
        時間複雜度:$O((HW + Q) \log HW)$ \\
        (多花點力氣 $O(HW + Q \log HW)$)
    }

    \only<2> {
        時間複雜度:$O(HW + \yum{16}Q \log HW)$ \\
        常數巨大!
    }
\end{frame}

\begin{frame}{\btitle{好的連續子序列}}
    \begin{problem}[好的連續子序列,台大演算法設計與分析(ADA)作業]
        給定一個 $1, 2, \dots, N$ 的排列,試求有多少子區間 $[l,r]$,滿足該子區間是一個連續正整數的排列?

        \begin{itemize}
            \item $1\le N \le 5\times10^5$
        \end{itemize}
    \end{problem}

    在 Codeforces 526F 有可以傳的 Judge
\end{frame}

\begin{frame}{\btitle{好的連續子序列}}
    跟一維的 Seats 比起來,少了修改,多了不是 $1$ 開頭的區間也要數數看

    \only<2-> {
        \begin{itemize}
            \item 用 Seats 作法找出 $1$ 開頭的好區間有幾個
            \item 把 $1$ 拿掉(讓他永遠是白色),找出 $2$ 開頭的好區間有幾個
            \item ......
            \item 把 $1, 2, \dots, N - 1$ 拿掉(讓他們永遠是白色),找出 $N$ 開頭的好區間有幾個
        \end{itemize}
    }

    \only<3> {
        題目沒叫你修改,但是你自己把「枚舉排列的開頭」當成 $N$ 次修改
    }
\end{frame}

\begin{frame}{\btitle{好的連續子序列}}
    \only<1> {
        時間複雜度:$O(N \log N)$
    }

    \only<2> {
        時間複雜度:\yum{至少 $O(4N \log N)$}

        常數巨大!我在 NEOJ 吃 TLE
    }
\end{frame}

\begin{frame}{\btitle{好的連續子序列:番外}}
    本題官方作法是分治,也有其他使用大資料結構但可以時限內通過的作法

    你能想到幾種不同的作法?
\end{frame}

\begin{frame}{\btitle{好的連續子序列:番外}}
    關鍵觀察:好區間 $\iff r - l = \max_{l \leq i \leq r}(a_i) - \min_{l \leq i \leq r}(a_i)$

    \only<2> {
        分治作法($O(N \log N)$)
    
        \begin{itemize}
            \item 序列切兩半,數跨兩邊的好的子序列
            \begin{itemize}
                \item 分四種情況:最大值和最小值分別在左半邊還是右半邊
                \item 時間複雜度 $O(N)$
            \end{itemize}
            \item 兩邊遞迴分治
        \end{itemize}
    }
    \only<3> {
        資料結構作法($O(N \log N)$)
    
        \begin{itemize}
            \item 掃描線枚舉區間右界,用資結維護每個左界對應到的 $\left(\left(\max - \min\right) - \left(r - l\right)\right)$
            \begin{itemize}
                \item $\max, \min$ 都可以單調隊列區間加值
                \item 數數看有幾個 $0$
                \item 這坨總是 $\geq 0$,可以數最小值個數
            \end{itemize}
        \end{itemize}
    }
\end{frame}

\begin{frame}{\ebtitle}
    \begin{itemize}
        \item 題目要維護的東西難以維護
        \item 找到小小的特徵點,用小特徵湊出題目要的條件
        \item 小特徵足夠單純可以維護
    \end{itemize}

    資結不是重點,重點是發現精妙的轉換和觀察
\end{frame}
